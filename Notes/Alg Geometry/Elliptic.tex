\documentclass{article}

\usepackage{../../header-colourful}
%%%%%%%%%%%%%%%%%%%%%%%%%%%%%%%%%%%%%%%%%%%%%%%%%%%%%%%%
%Preamble

\title{Elliptic functions, integrals, and curves}
\author{Linden Disney-Hogg}
\date{June 2020}

%%%%%%%%%%%%%%%%%%%%%%%%%%%%%%%%%%%%%%%%%%%%%%%%%%%%%%%%
%%%%%%%%%%%%%%%%%%%%%%%%%%%%%%%%%%%%%%%%%%%%%%%%%%%%%%%%
\begin{document}

\maketitle
\tableofcontents

%%%%%%%%%%%%%%%%%%%%%%%%%%%%%%%%%%%%%%%%%%%%%%%%%%%%%%%%
%%%%%%%%%%%%%%%%%%%%%%%%%%%%%%%%%%%%%%%%%%%%%%%%%%%%%%%%
\section{Introduction}
Elliptic integrals are hard to solve and make little sense. Fortunately, algebraic geometry can help us. These will be my personal notes which will accumulate many resources, which I will try to reference, though I doubt I will give when each one was used. The current list is 
\begin{itemize}
	\item \textit{Elliptic Funcations and Applications} (Lawden)
	\item \textit{Elliptic Curves} (McKean, Moll) 
	\item often wikipedia
	\item nlab
	\item \textit{Algebraic Curves and Riemann Surfaces} (Miranda)
	\item \textit{Arithmetic of Elliptic Curves} (Silverman)
\end{itemize}
These books may use different conventions, but I will keep mine uniform (\hl{unless I make an error, in which case please do let me know}). 

%%%%%%%%%%%%%%%%%%%%%%%%%%%%%%%%%%%%%%%%%%%%%%%%%%%%%%%%
%%%%%%%%%%%%%%%%%%%%%%%%%%%%%%%%%%%%%%%%%%%%%%%%%%%%%%%%
\section{Preliminaries}
We should hopefully not need too many preliminaries in these notes, but those that we do will form two main categories: algebraic geometry and Riemann surfaces. I desire to eventually make notes about these subjects, but in the interim this will suffice. 

%%%%%%%%%%%%%%%%%%%%%%%%%%%%%%%%%%%%%%%%%%%%%%%%%%%%%%%%
\subsection{Complex Analysis}

I will fix some notation with a definition:

\begin{definition}
	Denote the \bam{Upper Half Plane (UHP)} in $\mbb{C}$ as 
	\eq{
H = \pbrace{z \in \mbb{C} \, | \, \Im z > 0}	
}
\end{definition}

Recall that for any distinct $z_1, z_2, z_3 \in \mbb{C}$ we have a Mobius transform that sends them to $0,1,\infty$ given by 
\eq{
f(z) = \frac{z-z_1}{z-z_3}\frac{z_2 - z_3}{z_2 - z_1}
}
\begin{definition}
Given distinct $z_0, z_1, z_2, z_3 \in \mbb{C}$ the \bam{cross ratio} is 
\eq{
\frac{z_0-z_1}{z_0-z_3}\frac{z_2 - z_3}{z_2 - z_1}
}
\end{definition}

\begin{prop}
Distinct $z_0, z_1, z_2, z_3 \in \mbb{C}$ lie on a circline iff their cross-ratio is real. 
\end{prop}

\begin{definition}
	The \bam{modular group} is 
	\eq{
\Gamma = PSL(2,\mbb{Z}) = \faktor{SL(2,\mbb{Z})}{\pbrace{\pm I}}	
}
\end{definition}

\begin{theorem}[Liouville]
	A bounded entire function $f:\mbb{C} \to \mbb{C}$ is constan. 
\end{theorem}

%%%%%%%%%%%%%%%%%%%%%%%%%%%%%%%%%%%%%%%%%%%%%%%%%%%%%%%%
\subsection{Algebraic Geometry}

\begin{definition}
	A \bam{field extension} if a pair of fields $K \subseteq L$ s.t. the operations of $K$ are those of the restriction of $L$. It is denoted $L / K$
\end{definition}

\begin{example}
	The set 
	\eq{
\mbb{Q}(\sqrt{2}) = \pbrace{a+b\sqrt{2} \, | \, a,b \in \mbb{Q}}	
}
is a field extension of $\mbb{Q}$. 
\end{example}

\begin{prop}
	If $L/K$ is a field extension, $L$ is \ $K$-vector space. 
\end{prop}

\begin{definition}
	The \bam{degree} of a field extension $L/K$ is the dimension of $L$ as a $K$-vector space. It is denoted $[L:K]$.
\end{definition}

\begin{example}
	$[\mbb{Q}(\sqrt{2}):\mbb{Q}] = 2$ as it has basis $\pbrace{1,\sqrt{2}}$
\end{example}

\begin{definition}
	The \bam{transcendence degree} of a field extension $L/K$ is the maximum cardinality of an algebraic independent subset of $L$ over $K$. 
\end{definition}

\begin{example}
	$\mbb{Q}(\sqrt{2},e)/\mbb{Q}$ has transcendence degree 1.
\end{example}

\hl{replace these examples with more C oriented ones. }


%%%%%%%%%%%%%%%%%%%%%%%%%%%%%%%%%%%%%%%%%%%%%%%%%%%%%%%%
\subsection{Riemann Surfaces}

%%%%%%%%%%%%%%%%%%%%%%%%%%%%%%%%%%%%%%%%%%%%%%%%%%%%%%%%
\subsubsection{Riemann Sphere}

%%%%%%%%%%%%%%%%%%%%%%%%%%%%%%%%%%%%%%%%%%%%%%%%%%%%%%%%
\subsubsection{Smooth Projective Plane Curves}

\begin{definition}
	A $\mbb{C}$-polynomial $F(\bm{x})=F(x_1, \dots, x_n)$ is \bam{homogeneous of degree d} if 
	\eq{
\forall \lambda \in \mbb{C}, \, F(\lambda \bm{x}) = \lambda^d F(\bm{x})	
}
\end{definition}

\begin{definition}
	The \bam{projective plane curve} defined by the homogeneous polynomial $F:\mbb{C}^3 \to \mbb{C}$ is the closed subset
	\eq{
X = \pbrace{[x:y:z] \in \mbb{P}^2 \, | \, F(x,y,z) = 0}	\subset \mbb{P}^2
}
\end{definition}

\begin{remark}
	Note that this is well defined as $F(x,y,z) = 0 \Leftrightarrow \lambda^d F(x,y,z)=F(\lambda x, \lambda y, \lambda z)=0$
\end{remark}

\begin{definition}
	A homogeneous polynomial $F:\mbb{C}^n \to \mbb{C}$ is \bam{singular} at $\bm{x}$ (i.e. $\bm{x}$ is a \bam{singular point}) if $\forall 1 \leq i \leq n$
	\eq{
\ev{\pd[F]{x_i}}{\bm{x}} = 0 	
}
The polynomial is said to be \bam{non-singular} if it has no singular points
\end{definition}

\begin{prop}
	The projective plane curve corresponding to a non-singular homogeneous polynomial is a compact Riemann surface. 
\end{prop}

%%%%%%%%%%%%%%%%%%%%%%%%%%%%%%%%%%%%%%%%%%%%%%%%%%%%%%%%
\subsubsection{Complex Tori}

\begin{definition}
	A map between Riemann surfaces $f:X \to Y$ is called an \bam{isomorphism} iff the induced maps on open subsets of $\mbb{C}$ is bijective and analytic. If there is an isomorphism between $X,Y$, they are said to be isomorphic.  
\end{definition}

\begin{definition}
	Fix $\omega_1, \omega_2 \in \mbb{C}$ linearly independent over $\mbb{R}$ and let 
	\eq{
L = L(\omega_1, \omega_2) = \mbb{Z}\omega_1 \oplus \mbb{Z}\omega_2 \subset \mbb{C}
}
The \bam{complex tori with lattice $\bm{L}$} is $X = \faktor{\mbb{C}}{L}$. The corresponding \bam{fundamental cell} is 
\eq{
\mf{F} = \mf{F}(X) = \pbrace{\alpha \omega_1+ \beta \omega_2 \, | \, \alpha,\beta \in [0,1)}
}
\end{definition}

\begin{prop}
	Complex tori are Riemann surfaces. 
\end{prop}

\begin{prop}
	Two complex tori with lattices $L(\omega_1, \omega_2), \, L(\omega_1^\prime, \omega_2^\prime)$ are isomorphic iff $\exists$ a fractional linear transformation sending $\frac{\omega_2}{\omega_1} \mapsto \frac{\omega_2^\prime}{\omega_1^\prime}$
\end{prop}

\begin{prop}
	Every complex tori can be written with a lattice the form $L(1,\tau)$ where $\Im\tau > 0$
\end{prop}

\begin{remark}
	From the above results we can view the orbits of the modular group acting on the UHP as a list of all possible complex structures on the topological torus. We will always assume $\frac{\omega_2}{\omega_1}$ has positive imaginary part.  
\end{remark}

%%%%%%%%%%%%%%%%%%%%%%%%%%%%%%%%%%%%%%%%%%%%%%%%%%%%%%%%
\subsubsection{Degree of a Map}

\begin{definition}
	Let $f:\mbb{C} \to \mbb{C}$ be meromorphic at $z_0 \in \mbb{C}$ about which it's Laurent series $\sum_n c_n (z-z_0)^n$. The \bam{order} of $f$ at $z_0$ is 
	\eq{
\ord_{z_0}(f) = \min\pbrace{n \, | \, c_n \neq 0}	
}
\end{definition}

\begin{remark}
	This definition naturally extends to maps between Riemann surfaces by taking local coordinates around the points. 
\end{remark}

\begin{lemma}
	Let $f,g : X \to \mbb{C}$ be non-zero meromorphic functions and $p  \in X$. Then 
	\begin{itemize}
		\item $\ord_p(fg) = \ord_p(f) + \ord_p(g)$
		\item $\ord_p\pround{\frac{1}{f}} = -\ord_p(f)$
		\item $\ord_p\pround{\frac{f}{g}} = \ord_p(f) - \ord_p(g)$
	\end{itemize}
\end{lemma}
\begin{proof}
	The first results follows from multiplying together the Laurent series. The second and third then follow. 
\end{proof}

\begin{lemma}
	Let $X$ be a Riemann surface and $f:X \to \mbb{C}$ a non-constant meromorphic function. The set
	\eq{
\pbrace{p \in X, \, \ord_p(f) \neq 0} \subset X	
} 
is discrete.
\end{lemma}
\begin{proof}
	If the set of poles of $f$ had an accumulation point, around this point $f$ would have no Laurent series. A similar argument gives no accumulation point for zeros. 
\end{proof}

\begin{corollary}
Let $X$ be a  compact Riemann surface and $f:X \to \mbb{C}$ a non-constant meromorphic function. The set
\eq{
	\pbrace{p \in X, \, \ord_p(f) \neq 0} \subset X	
}
is finite.
\end{corollary}

\begin{definition}
	Let $F:X \to Y$ be a map between Riemann surfaces holomorphic at $p \in X$. The \bam{multiplicity} of $F$ at $p$ is $\mult_p(F)=m$ the unique integer s.t. in local coordinates around $p$ $F$ is represented by 
	\eq{
	h(z) = h(z_0) + \sum_{i \geq m} c_i (z-z_0)^i
}
\end{definition}

\begin{definition}
Let $F:X \to Y$ be a non-constant holomorphic map of Riemann surfaces. $p \in X$ is called a \bam{ramification point} if $\mult_p(F) \geq 2$. $y\in Y$ is called a \bam{branch point} if $y=F(p)$ for some ramification point $p$. 
\end{definition}

\begin{remark}
	The thing to note here is that, generically, $\mult_p(F) = 1$ 
\end{remark}

\begin{prop}
	Let $f:X \to \mbb{C}$ be a meromorphic function and $F:X \to \mbb{C}_\infty$ the associated holomorphic map. Then 
	\begin{enumerate}
		\item If $p \in X$ is a zero of $f$, $\mult_p(F) = \ord_p(f)$
		\item If $p \in X$ is a pole of $f$, $\mult_p(F) = -\ord_p(f)$
		\item If $p \in X$ is neither a zero or a pole, $\mult_p(F) = \ord_p(f-f(p))$
	\end{enumerate}
\end{prop}

\begin{prop}
	Let $F:X \to Y$ be a holomorphic non-constant map of Riemann surfaces and consider the map $d_{\cdot}(F):Y \to \mbb{Z}$ 
	\eq{
d_y(F) = \sum_{p \in F^{-1}(y)} \mult_p(F)	
}
The map $d$ is constant. 
\end{prop}

\begin{definition}
Let $F:X \to Y$ be a holomorphic non-constant map of Riemann surfaces. The \bam{degree} of $F$ is 
\eq{
\deg(F) = d_y(F) \text{ for some $y \in Y$}
}
\end{definition}

\begin{prop}
	Let $X$ be a compact Riemann surface and $f:X \to \mbb{C}$ a non-constant meromorphic function. Then 
	\eq{
\sum_{p \in X} \ord_p(f) = 0	
}
\end{prop}
\begin{proof}
	Consider the corresponding map $F: X \to \mbb{C}_\infty$. Let $\pbrace{x_i} = F^{-1}(0), \ ,\pbrace{y_j} = F^{-1}(\infty)$. Then 
	\eq{
\sum_p \ord_p(f) &= \sum_i \ord_{x_i}(f) + \sum_j \ord_{y_j}(f) \\
&= \sum_i \mult_{x_i}(F) - \sum_j \mult_{y_j}(F) \\
&= d_0(F) - d_\infty(F) = 0	
}
\end{proof}

\begin{remark}
	As $f$ can only have finitely many poles and zeros, for all but finitely many $p \in X$ we have $\ord_p(f) = 0$ so the sum makes sense. 
\end{remark}

\begin{theorem}[Hurwitz Formula]
	Let $F:X \to Y$ be a non-constant holomorphic map between compact Riemann surfaces. Then 
	\eq{
g(X) - 1 = \deg(F) \psquare{g(Y)-1} + \frac{1}{2} \sum_{p \in X} \psquare{\mult_p(F)-1}	
}
where $g$ is the genus of the underlying topological surface. 
\end{theorem}

\begin{example}
	An example that will be important to example later is that of a projective plane curve $X$ given by 
	\eq{
X = \pbrace{[x:y:z] \, | \, P(x,y,z)=0} \subset \mbb{P}^2	
}
where $F: \mbb{C}^3 \to \mbb{C}$ is the homogeneous polynomial 
\eq{
P(x,y,z) = y^2z - x^3 + xz^2 -z^3
}
We note we can write 
\eq{
X = \pbrace{[x:y:1] \, | \, y^2 = x^3-x+1} \cup \pbrace{[0:1:0]}
}
We consider a map $F:X \to \mbb{P}^1$ by taking a coordinate on the curve. On the patch $z \neq 0$ we can choose $x$, and around $z=0=x$ we can choose $1/x$. We find that a generic point has two preimages except at the roots of $x^3-x+1$ or at 'infinity' ($z=0=x$) as at all these points there is only one value of $y$. Hence 
\eq{
g(X) = 1 + 2\psquare{0-1} + \frac{1}{2} \times 4\psquare{2-1} = 1
}
so topologically this curve is a torus.
\end{example}

%%%%%%%%%%%%%%%%%%%%%%%%%%%%%%%%%%%%%%%%%%%%%%%%%%%%%%%%
\subsubsection{Field of Functions}

\begin{definition}
	Let $X$ be a Riemann surface. The set of all meromorphic functions $X \to \mbb{C}$ is called the \bam{field of functions} of $X$ and is denoted $K(X)$. 
\end{definition}

\begin{example}
	$K(\mbb{P}^1) \cong \mbb{C}(x)$
\end{example}

\begin{prop}
	$K(X)$ is a field extension of $\mbb{C}$. 
\end{prop}

\begin{prop}
	If $X$ is a compact Riemann surface then $K(X)/\mbb{C}$ is a field extension of transcendence degree 1.  
\end{prop}

\begin{definition}
	Let $X, Y$ be Riemann surfaces and $p : X \to Y$ a branched covering map. A \bam{deck
	transformation} is a fiber preserving biholomorphic map, that is, a map f such that the diagram
	\begin{tkz}
		X \arrow[r,"f"] \arrow[d,"p"'] & X \arrow[dl,"p"]\\
		Y &
	\end{tkz}
	commutes. We denote the set of them  $\text{Deck}(X/Y)$
\end{definition}

\begin{prop}
	The set of deck transformations form a group under composition
\end{prop}

\begin{theorem}
	Let $X,Y$ be compact Riemann surfaces and $p:X \to Y$ a $n$-fold branched covering. Then $K(X)/p^\ast K(Y)$ is a degree-$n$ field extension. \\
	Conversely let $Y$ be a Riemann surface and $L/K(Y)$ a degree-$n$ field extension. Then $\exists X$ a Riemann surface, $p:X \to Y$ an $n$-sheeted branched covering map, and $f \in K(X)$ s.t. $L\cong K(X)=p^\ast K(Y)(f)$. \\
	In both cases, $\text{Deck}(X/Y) \cong \Aut(K(X) / p^\ast K(Y))$  
\end{theorem}

\begin{corollary}
	Let $X,Y$ be compact Riemann surfaces, then $X\cong Y$ iff $K(X) \cong K(Y)$. 
\end{corollary}

\begin{remark}
	We can view this final remark from the point of view of category theory. We can let $R$ be the category of compact Riemann surfaces with morphisms given by holomorphic maps. We then give the contravariant functor 
	\eq{
K:R \to \text{Sets}	
}
sending a compact Riemann surface $X$ to its function field $K(X)$, and sending morphisms to their pullback. This is a representable functor as 
\eq{
K(X) = \Hom(X,\mbb{P}^1)
}
Hence that $X \cong Y \Leftrightarrow K(X) \cong K(Y)$ is a consequence of the Yoneda lemma. 
\end{remark}

%%%%%%%%%%%%%%%%%%%%%%%%%%%%%%%%%%%%%%%%%%%%%%%%%%%%%%%%
%%%%%%%%%%%%%%%%%%%%%%%%%%%%%%%%%%%%%%%%%%%%%%%%%%%%%%%%
%%%%%%%%%%%%%%%%%%%%%%%%%%%%%%%%%%%%%%%%%%%%%%%%%%%%%%%%
%%%%%%%%%%%%%%%%%%%%%%%%%%%%%%%%%%%%%%%%%%%%%%%%%%%%%%%%
\part{Functions}

%%%%%%%%%%%%%%%%%%%%%%%%%%%%%%%%%%%%%%%%%%%%%%%%%%%%%%%%
%%%%%%%%%%%%%%%%%%%%%%%%%%%%%%%%%%%%%%%%%%%%%%%%%%%%%%%%
\section{Periodic Functions}
We will start with some general results about periodic complex functions. We fix $f:\mbb{C} \to \mbb{C}$ to be a non-constant meromorphic single valued function. 

\begin{lemma}
	The periods of $f$ form a $\mbb{Z}$-module $L$ 
\end{lemma}

\begin{prop}
	$L$ is either 
	\begin{itemize}
		\item trivial, ($=0$)
		\item rank 1, ($=\omega \mbb{Z}$)
		\item rank 2, ($=\omega_1 \mbb{Z} \oplus \omega_2 \mbb{Z}$)
	\end{itemize}
\end{prop}

\begin{definition}
	A function with rank-2 $L$ is called an \bam{elliptic function} and $L$ is called the \bam{period lattice}. 
\end{definition}

\begin{definition}
	Given a period lattice $L$, a choice of $\omega_1, \omega_2$ s.t.  $L = \omega_1 \mbb{Z} \oplus \omega_2 \mbb{Z}$ is called a \bam{primitive pair}. For a primitive pair we call $\tau = \frac{\omega_2}{\omega_1}$ the \bam{period ratio}. 
\end{definition}

It is common, as we will do now, to restrict the period ratio to have real imaginary part

\begin{prop}
	Any pairs of primitive pairs are related by a fractional linear transform $\in SL(2,\mbb{R})$. The corresponding period ratios are related by a map in the modular group
\end{prop}

\begin{definition}
	An \bam{elliptic function field} is the class $K =K(X)$ of functions of rational character on the complex torus $X$
\end{definition}

\begin{remark}
	Idefintifying $X$ with $\faktor{\mbb{C}}{L}$ is equivalent to identifying $K(X)$ with double periodic functions on the universal cover of $X$, $\mbb{C}$.
\end{remark}

\begin{prop}
	$K(X)$ is a differential field with differential $\frac{d}{dz}$ inherited from the coordinate $z$ on the universal cover.  
\end{prop}

\begin{prop}
$\deg f = 0$ iff $f$ is constant
\end{prop}
\begin{proof}
	$\deg(f) = 0$ iff $f$ has no poles. Pulling back to a doubly periodic function on $\mbb{C}$, it is a bounded entire function, so constant by Liouville's theorem. 
\end{proof}

\begin{lemma}
	Let $f \in K(X)$. Then 
	\eq{
\oint_{\del \mf{F}(X)} f(z) \, dz = 0 	
}
\end{lemma}
\begin{proof}
	Use periodicity of the function. 
\end{proof}

\begin{remark}
	This can be used to show $\sum_{p\in X} \ord_p(f)=0$ for $X$ a complex torus, by considering the integral of $\frac{1}{2\pi i}\frac{f^\prime}{f} \in K(X)$
\end{remark}

This result has some powerful corollaries, which we will see here now:

\begin{prop}
	$\forall f \in K(X), \, \deg f \neq 1$.
\end{prop}
\begin{proof}
	If $\deg f = 1$ then the residue at the pole is 
	\eq{
\frac{1}{2\pi i} \oint_{\del \mf{F}(X)} f(z) \, dz = 0	
}
so there is no pole. 
\end{proof}

\begin{prop}
Take $f \in K(X)$ and let $p_1, \dots, p_d$ be the zeros, $q_1, \dots, q_d$ the poles. Then 
\eq{
\sum p_i  - q_i \in L
} 
\end{prop}
\begin{proof}
Let $L = L(\omega_1, \omega_2)$
\eq{
\sum p_i  - q_i &= \frac{1}{2\pi i}\oint z d(\log f(z)) \\
&= \frac{1}{2\pi i} \pbrace{\int_0^{\omega_1} \psquare{z - (z+\omega_2)}d(\log f(z)) - \int_0^{\omega_2} \psquare{z - (z+\omega_1)} d(\log f(z)) } \\
&= \omega_1 \underbrace{\psquare{\frac{1}{2\pi i}\int_0^{\omega_2} d(\log f(z))}}_{\in \mbb{Z}} - \omega_2  \underbrace{\psquare{\frac{1}{2\pi i}\int_0^{\omega_1} d(\log f(z))}}_{\in \mbb{Z}}
}
where we know the integrals are integer values as $f$ takes the same values at the corners, so the argument must differ by an element of $2\pi i \mbb{Z}$
\end{proof}

%%%%%%%%%%%%%%%%%%%%%%%%%%%%%%%%%%%%%%%%%%%%%%%%%%%%%%%%
\subsection{Automorphisms}

\begin{prop}
	Complex torus $X$ always have bijective automorphisms $z \mapsto z+c, \, z \mapsto -z$. The only involutions are the reflection $z \mapsto -z$ and addition of half periods. 
\end{prop}

\begin{remark}
	Note that every complex torus $X$ admits the bijective automorphism descending from the map $\mbb{C} \to \mbb{C}, \ ,z \mapsto n z$ for $n \in \mbb{Z}\setminus 0$
\end{remark}

\begin{definition}
	The complex torus $X$ with primitive periods $\omega_1, \omega_2$ is said to \bam{admit complex multiplication} if $\exists f$ an automorphism $f(z) = cz$ for some $c \in \mbb{C}\setminus\mbb{Z}$. In this we have $i,j,k,l \in \mbb{Z}$ s.t.
	\eq{
c \omega_2 &= i \omega_2 + j \omega_1 \\
c \omega_1 &= k \omega_2 + l \omega_1	
}
i.e. $\frac{\omega_2}{\omega_1}$ is a fixed points of the fractional linear transform $\begin{psmallmatrix} i & j \\ k & l \end{psmallmatrix}$. 
\end{definition}

\begin{lemma}
	The period ratio is a quadratic irrationality from the field $\mbb{Q}[\sqrt{(i+l)^2 - 4(il-jk)}]$
\end{lemma}

\begin{prop}
	Suppose $X$ is a complex torus admiitting complex multiplication via a bijective automorphisms, then the period ratio is either $i$ or $e^{\frac{\pi i}{3}}$. 
\end{prop}


%%%%%%%%%%%%%%%%%%%%%%%%%%%%%%%%%%%%%%%%%%%%%%%%%%%%%%%%
%%%%%%%%%%%%%%%%%%%%%%%%%%%%%%%%%%%%%%%%%%%%%%%%%%%%%%%%
\section{Theta Functions}
%%%%%%%%%%%%%%%%%%%%%%%%%%%%%%%%%%%%%%%%%%%%%%%%%%%%%%%%
\subsection{Definitions}
We start this section with a warning. There are \emph{many} different ways of writing theta functions, and so you will need to be on your toes to connect those in these notes with those in other documents. These will start with the definition in 

\begin{definition}
	For $\tau \in \mbb{C}, \, \image(\tau)>0$, define the \bam{Riemann theta function} 
	\eq{
\theta(z) = \theta(z|\tau) = \sum_{n \in \mbb{Z}} e^{\pi i\psquare{n^2 \tau + 2nz}}	
}
\end{definition}

\begin{remark}
	This is the definition given in Miranda and on Wikipedia (as of June 2020). 
\end{remark}

\begin{prop}
	The series converges absolutely and uniformly on compact subset of $\mbb{C}$. Hence $\theta$ is an analytic function on $\mbb{C}$. 
\end{prop}

\begin{prop}
	We have 
	\eq{
\theta(z+1) &= \theta(z) \\
\theta(z+\tau) &= e^{-\pi i\psquare{\tau +2z}}\theta(z)
}
\end{prop}

\begin{corollary}
	$\theta(z_0) \Leftrightarrow \forall m,n \in \mbb{Z}, \, \theta(z_0 + m + n\tau)=0$ and the order of the zero is the same
\end{corollary}

\begin{prop}
	The only zeros of $\theta$ are 
	\eq{
\frac{1+\tau}{2} + L(1,\tau)	
}
and all these zeros are simple. 
\end{prop}

\begin{definition}
	The \bam{translated theta function} is 
	\eq{
\theta^{(x)}(z) = \theta\pround{z - (\sfrac{1}{2}) - (\sfrac{\tau}{2}) - x}	
}
\end{definition}

\begin{definition}
	The \bam{Jacobi theta functions} are 
	\eq{
\theta_1(z | \tau) = 	
}
\end{definition}
%%%%%%%%%%%%%%%%%%%%%%%%%%%%%%%%%%%%%%%%%%%%%%%%%%%%%%%%
%%%%%%%%%%%%%%%%%%%%%%%%%%%%%%%%%%%%%%%%%%%%%%%%%%%%%%%%
\section{Jacobi Elliptic Functions}

\begin{definition}
	The \bam{sinus amplitudinus} function is $sn:\mbb{C} \to \mbb{C}$ given by 
	\eq{
	x = \int_0^{sn(x,k)} \frac{dy}{\sqrt{(1-y^2)(1-k^2 y^2)}}
}
\end{definition}

%%%%%%%%%%%%%%%%%%%%%%%%%%%%%%%%%%%%%%%%%%%%%%%%%%%%%%%%
%%%%%%%%%%%%%%%%%%%%%%%%%%%%%%%%%%%%%%%%%%%%%%%%%%%%%%%%
\section{\secmath{\text{The Weierstrass $\wp$ Function}}}

From our discussion of doubly-periodic functions, we know the minimal degree of a non-constand function is 2, so the simplest possible cases are 
\begin{itemize}
	\item two simple poles
	\item one double pole
\end{itemize}
We will construct the latter case.

\begin{definition}
	Pick a complex torus $X$ with lattice $L$. The corresponding \bam{Weierstrass function} is $\wp:\mbb{C} \to \mbb{C}$ 
	\eq{
\wp(z) = \wp(z | L) = \frac{1}{z^2} + \sum_{\omega \in L\setminus 0} \psquare{\frac{1}{(z-\omega)^2} - \frac{1}{\omega^2}}
} 
\end{definition}

\begin{prop}
	The infinite sum is convergent and hence the $\wp$ function is well defined.
\end{prop}

\begin{prop}
	$\wp$ has the following properties:
	\begin{itemize}
		\item Even, i.e. $\wp(z) = \wp(-z)$.
		\item Double periodic with period lattice $L$
		\item $\deg \wp = 2$ as a map $\faktor{\mbb{C}}{L} \to \mbb{P}^1$, with only one double pole at $z=0$
		\item $\forall c \in \mbb{C}^\times, \, \wp(cz | cL) = \wp(z | L)$
		\item $\wp^\prime(z) = -2\sum_{\omega \in L} (z-\omega)^{-3}$
		\item $\wp^\prime$ is odd
		\item $\forall \omega \in L, \, \wp^\prime(\sfrac{\omega}{2}) =0$ and the half periods are the only roots. 
		\item Let $\omega_1, \omega_2$ be the primitive roots and $e_1 = \wp(\sfrac{\omega_1}{2}), \, e_2 = \wp(\sfrac{\omega_1}{2} + \sfrac{\omega_2}{2}), \, e_3 = \wp(\sfrac{\omega_2}{2})$. Then $e_1, e_2, e_3$ are distinct.
	\end{itemize}
\end{prop}

\begin{prop}
	$\wp$ satisfies the differential equations 
	\eq{
\pround{\wp^\prime}^2 &= 4(\wp-e_1)(\wp-e_2)(\wp-e_3) \\
&= 4\wp^3 -g_2 \wp -g_3	
}
where 
\eq{
g_2  &= 60\sum_{\omega \in L \setminus 0} \omega^{-4} \\
g_3 &= 140\sum_{\omega \in L \setminus 0} \omega^{-6}
}
are the \bam{invariant of the cubic}
\end{prop}

\begin{remark}
	It is shown in the proof of the above proposition that 
	\eq{
g_2 &= -4(e_1 e_2 + e_2 e_3 + e_3 e_1) \\
g_3 &= 4e_1 e_2 e_3	
}
and 
\eq{
e_1 + e_2 + e_3 = 0
}
\end{remark}

\begin{remark}
We will sometimes use the notation $\wp(z) = \wp(z | g_2, g_3)$
\end{remark}

\begin{lemma}
	$g_2^3 - 27g_3^2 = 16(e_1 - e_2)^2(e_2 - e_3)^2 (e_3 - e_1)^2$
\end{lemma}
\begin{proof}
	Consider the polynomial $y^2 = x^3 - \pround{\sfrac{g_2}{4}}x - \pround{\sfrac{g_3}{4}} = (x-e_1)(x-e_2)(x-e_3)$. The discriminant of this cubic is 
	\eq{
\Delta &= 4 \pround{\sfrac{g_2}{4}}^3 - 27\pround{\sfrac{g_3}{4}}^2 \\
&= \frac{1}{16} \pround{g_2^3 - 27 g_3^2}
}
However, we also know 
\eq{
\Delta = \prod_{i < j} (e_i - e_j)^2
}
\end{proof}


\begin{corollary}
	$\wp^{\prime \prime} = 6\wp^2 - \frac{1}{2}g_2
$\end{corollary}

%%%%%%%%%%%%%%%%%%%%%%%%%%%%%%%%%%%%%%%%%%%%%%%%%%%%%%%%
\subsection{Addition Theorems}

\begin{prop}
	$\wp(z+w) = \frac{1}{4} \psquare{\frac{\wp^\prime(z) - \wp^\prime(w)}{\wp(z) - \wp(w)}}^2 - \wp(z) - \wp(w)$
\end{prop}
\begin{corollary}
	$\wp(z + \sfrac{\omega_1}{2}) = e_1 + \frac{(e_1 - e_2)(e_1 - e_3)}{\wp(z) - e_1}$ and similar equations for cycles 
\end{corollary}
\begin{proof}
	Start by noting 
	\eq{
\wp(z+\sfrac{\omega_1}{2}) + \wp(z) + e_1 = \frac{1}{4} \psquare{\frac{\wp^\prime(z)}{\wp(z)-e_1}}^2	
}
\end{proof}


\begin{corollary}[Duplication Formula]
	$\wp(2z) = \frac{1}{4} \psquare{\frac{\wp^{\prime \prime}(z)}{\wp^\prime(z)}}^2 - 2\wp(z)$
\end{corollary}
%%%%%%%%%%%%%%%%%%%%%%%%%%%%%%%%%%%%%%%%%%%%%%%%%%%%%%%%
%%%%%%%%%%%%%%%%%%%%%%%%%%%%%%%%%%%%%%%%%%%%%%%%%%%%%%%%
%%%%%%%%%%%%%%%%%%%%%%%%%%%%%%%%%%%%%%%%%%%%%%%%%%%%%%%%
%%%%%%%%%%%%%%%%%%%%%%%%%%%%%%%%%%%%%%%%%%%%%%%%%%%%%%%%
\part{Integrals}

%%%%%%%%%%%%%%%%%%%%%%%%%%%%%%%%%%%%%%%%%%%%%%%%%%%%%%%%
%%%%%%%%%%%%%%%%%%%%%%%%%%%%%%%%%%%%%%%%%%%%%%%%%%%%%%%%
\section{Elliptic integrals}


%%%%%%%%%%%%%%%%%%%%%%%%%%%%%%%%%%%%%%%%%%%%%%%%%%%%%%%%
\subsection{Elliptic Integral of the First Kind}
Let us start by stating our assumptions of this section, namely:
\begin{itemize}
	\item $u \in \mbb{R}$
	\item $k \in (0,1)$.
\end{itemize}

We make the following definition:

\begin{definition}
	\bam{Jacobi's incomplete elliptic integral of the first kind} is the map $H\to \mbb{C}$
	\eq{
x \mapsto \int_0^x \frac{1}{\sqrt{(1-t^2)(1-k^2t^2)}} dt	
}
where the integrand is taken to be real and positive for $t \in (-1,1)$, and at the branch points $\pm 1, \pm \frac{1}{k}$ the contour should take an infinitesimal circle above the point (i.e remaining in the UHP). 
\end{definition}

\begin{definition}
	The \bam{complete elliptic integral of the first kind } is 
	\eq{
		K = K(k) = \int_0^1 \frac{dt}{\sqrt{(1-t^2)(1-k^2t^2)}} \in \mbb{R}
	}
	and the \bam{complementary integral of the first kind } is 
	\eq{
		K^\prime = K^\prime(k) = \int_1^\frac{1}{k} \frac{dt}{\sqrt{(t^2-1)(1-k^2t^2)}} \in \mbb{R}
	}
\end{definition}

\begin{prop}
	We have the following results about $K(k)$:
	\begin{enumerate}
		\item $K(-k) = K(k)$
		\item $\lim_{k \to 0^+} K(k) = \frac{\pi}{2}$
		\item $K(ik) = \frac{1}{\sqrt{1+k^2}}K\pround{\frac{k}{\sqrt{1+k^2}}}$
		\item $K(i) = \frac{1}{4}B\pround{\frac{1}{4},\frac{1}{2}}$ ($B$ the beta function)
		\item $K^\prime(k) = K(k^\prime)$ where $k^\prime = \sqrt{1-k^2}$
		\item $K(k) = (1+k_1) K(k_1)$ for $k_1 = \frac{1-k^\prime}{1+k^\prime}$
		\item $K(k) = \frac{\pi}{2} \prod_{n=1}^\infty (1+k_n)$ for $k_{n+1} = \frac{1-k_n^\prime}{1+k_n^\prime}$
		\item 	$K(k)  = \frac{1}{1+k} K \pround{\frac{2\sqrt{k}}{1+k}}$ (Landen's Transformation)
	\end{enumerate}
\end{prop}
\begin{proof}
	We will complete the proof in parts:
\begin{enumerate}
	\item Trivial 
	\item say something about convergence, and then $\int_0^1 \frac{dt}{\sqrt{1-t^2}} = \frac{\pi}{2}$ is a trig integral. 
	\item  consider the transform $	t \mapsto \frac{t}{\sqrt{1+k^3(1-t^2)}}$
	\item $K(i) = \int_0^1 \frac{dt}{\sqrt{1-t^4}} = \frac{1}{4} \int_0^1 t^{-\frac{3}{4}}(1-t)^{-\frac{1}{2}} \, dt$ making the substitution $t\mapsto t^\frac{1}{4}$.
	\item Use the substitution $s = \psquare{1-(k^\prime)^2 t^2}^{-\frac{1}{2}}$ to transform $K(k^\prime)$ to $K^\prime(k)$.
	\item do the calculation
	\item Iterate, observe $k_{n+1} < k_n$ (one can show $k_{n+1}<k_n^2$), and argue about convergence. 
	\item (Legendre 1811) Using the substitution $t \mapsto \frac{(1+k^\prime)x\sqrt{1-t^2}}{1-k^2 t^2}$ find $K(k_1) = \frac{1+k^\prime}{2}K(k)$ and the use Landen's transform.  
\end{enumerate}
\end{proof}


These definitions of the complete integral make sense for the following result:

\begin{prop}
	The incomplete elliptic integral of the first kind has the following straight line segements for ranges:
	\eq{
\psquare{0,1} &\mapsto [0,K]  & [-1,0] &\mapsto [-K,0] \\
\psquare{1,\sfrac{1}{k}} &\mapsto [K,K+iK^\prime] & \psquare{-\sfrac{1}{k},1} &\mapsto \psquare{-K+iK^\prime,-K}\\
\left[\sfrac{1}{k},\infty\right) &\mapsto [K+iK^\prime,iK^\prime)	& \left( -\infty, -\sfrac{1}{k}\right] &\mapsto (iK^\prime,-K+iK^\prime]
}
\end{prop}

\begin{corollary}
	The image of the UHP under the elliptic integral of the first kind is the interior of the rectangle with corners $\pbrace{K,-K,K+iK^\prime, -K+iK^\prime}$, with the boundary of the rectangle being the image of the boundary of the UHP.
\end{corollary}

\begin{definition}
	The \bam{incomplete elliptic integral of the second kind} is the map $H\to \mbb{C}$ \eq{
x \mapsto \int^x_0 \sqrt{\frac{1-k^2 t^2}{1-t^2}} dt 	
}
using the contour above branch points.
\end{definition}

\begin{definition}
	The \bam{complete elliptic integral of the first kind } is 
	\eq{
		E = E(k) = \int_0^1 \sqrt{\frac{1-k^2t^2}{1-t^2}}\, dt \in \mbb{R}
	}
	and the \bam{complementary integral of the first kind } is 
	\eq{
		E^\prime = E^\prime(k) =  \int_1^\frac{1}{k} \sqrt{\frac{1-k^2t^2}{t^2-1}}\, dt \in \mbb{R}
	}
\end{definition}

\begin{prop}
	We have the following results about $E(k)$:
	\begin{enumerate}
		\item $E^\prime(k) = K(k^\prime) - E(k^\prime)$
		\item (Legendre 1825) $KE^\prime + EK^\prime - KK^\prime = \frac{\pi}{2}$
	\end{enumerate}
\end{prop}

\begin{definition}
	The \bam{incomplete elliptic integral of the third kind} is the map $H \to \mbb{C}$ 
	\eq{
x \mapsto \int^x \frac{1}{(t^2-c^2)\sqrt{(1-t^2)(1-k^2t^2)}}	dt
}
using the contour above branch points.
\end{definition}

%%%%%%%%%%%%%%%%%%%%%%%%%%%%%%%%%%%%%%%%%%%%%%%%%%%%%%%%
\subsection{Reduction of Elliptic Integrals}

\begin{definition}
	If $R=R(x,y)$ is a rational function in $x,y$ and $y^2=Q(x)$ is a polynomial of degree $d=3,4$ with distinct roots then the integral
	\eq{
\int R(x,y) \, dx	
}
is called a \bam{general elliptic integral}. 
\end{definition}

\begin{remark}
	Dealing with integrals such as the above when $d=1,2$ or there are repeated roots are treated in your earlier calculus classes, for example:
	\begin{itemize}
		\item $d=1$: write $y^2 = ax+b$ and make the substitution 
		\eq{
	x \mapsto \frac{1}{a}(x^2-b)	
	}
		\item $d=2$: write $y^2 = c(x-a)(x-b)$ and make the substitution 
		\eq{
	x \mapsto a = \frac{1}{4}(b-a) \pround{x - \frac{1}{x}}^2	
	}
	\end{itemize}
\end{remark}

\begin{remark}
	Depending on how the layout of this document ends up being, I may have a the section on Jacobi elliptic functions first. In this case, one would see that the incomplete integral of the first kind corresponds to the inverse of the function $\sn$. We could also generate integrals from the inverses of $\cn, \dn$. However from theory of elliptic functions, we know we can write the inverses of these in terms of $\sn^{-1}$, so transforms must exists for their corresponding integrals to reduce them to "known" forms. The same idea applies to the reduction of general elliptic integrals and we will see that now. 
\end{remark}

\begin{prop}
A general elliptic integral can be written as 
\eq{
\int \psquare{y^{-1}R_1(x) + R_2(x)}\, dx 
}	
\end{prop}
As a results of this, because any rational integral can be solved through partial fractions, in order to study general elliptic integrals we need know only how to deal with integrals of the form 
\eq{
\int \frac{R(x)}{y} dx
}
Let us start to deal with these:
\subsubsection{\secmath{d=3}}
Write 
\eq{
Q(x) = (x-e_0)(x-e_1)(x-e_2)
}
with $e_0,e_1,e_2$ distinct. Make the substitution 
\eq{
x \mapsto x^2 + e_0
}
sending 
\eq{
\frac{dx}{y} \mapsto \frac{2dx}{\sqrt{(x^2 + e_0 - e_1)(x^2 + e_0 - e_2)}}
}
The new $Q$ has roots $\pm \sqrt{e_1 - e_0}, \, \pm\sqrt{e_2 - e_0}$, so wlog we may take $d=4$.
\subsubsection{\secmath{d=4}}
We make now start procedurally:
\begin{enumerate}
	\item Write $Q(x) = (x-e_0)(x-e_1)(x-e_2)(x-e_3)$
	\item Send $e_0, \dots, e_3$ to $\pm1, \pm \frac{1}{k}$ using a fractional linear transform. This sends $Q(x) \mapsto (1-x^2)(1-k^2 x^2)$. This is possible if the cross ratio is 
	\eq{
	\frac{e_0 - e_1}{e_0 - e_3}\frac{e_2 - e_3}{e_2 - e_1} = \frac{4k}{(1+k)^2}
}
The distinctness of the $e$ means that the cross ratio is not $0,1,\infty$ and so $k \neq 0, \pm 1$. \hl{There is clearly a choice involved in k here, can it be shown that this choice does not matter, for example in simple cases?}
\begin{remark}
	This value of the cross ratio can be calculated from the map that sends $-1 ,\frac{1}{k}, -\frac{1}{k} \mapsto 0,1,\infty$. This is the map 
	\eq{
z \mapsto \frac{z+1}{z + \sfrac{1}{k}} \frac{\sfrac{2}{k}}{1+ \sfrac{1}{k}}	
}
This corresponds to mapping the roots as said above. We could consider a different way of mapping the roots, for example $e_0, e_1, e_2, e_3 \mapsto \frac{1}{k^\prime},-1,1,-\frac{1}{k^\prime}$. The symbol $k^\prime$ is used as the cross ratio for this change is now found from the map 
\eq{
z \mapsto \frac{z+1}{z+\sfrac{1}{k^\prime}} \frac{1+\sfrac{1}{k^\prime}}{2}
}
i.e. the cross ratio is 
\eq{
\frac{(1+k^\prime)^2}{4k^\prime}
}
We can relate $k^\prime$ to $k$ as 
\end{remark}
\item Write $R(x) = R_1(x^2) + xR_2(x^2)$ and note 
\eq{
\int \frac{xR_2(x^2)}{\sqrt{(1-x^2)(1-k^2x^2)}} dx  = \frac{1}{2} \int \frac{R_2(x^2)}{\sqrt{(1-x^2)(1-k^2x^2)}} d(x^2)
}
Hence this term reduces to previously solved cases ($d=2$). As such we are reduced to the case of
\eq{
\int \frac{R(x^2)}{y} dx
}
for $y^2 = (1-x^2)(1-k^2 x^2)$. 
\item Use the identities 
\eq{
\frac{x^2-a}{x^2-b} &= 1 + \frac{b-a}{x^2-b} \\
\frac{1}{(x^2-a)(x^2-b)} &= (a-b)^{-1}\pround{\frac{1}{x^2-a} - \frac{1}{x^2-b}}
}
to reduce to the classes of integrals
\eq{
I_n &= \int \frac{x^{2n}}{y} \, dx  \quad (n \geq 0) \\
I_n^\prime &= \int \frac{(x^2-c^2)^n}{y} \, dx \quad (n < 0)
}
for $y^2 = (1-x^2)(1-k^2 x^2)$.
\item Use relation 
\eq{
n I_{n-1} -n(1+k^2)I_n + (n+1)k^2 I_{n+1} -\frac{1}{2}(1+k^2)I_n = x^{2n}y + \text{cst}
}
to reduce $I_n$ to $I_0$ and $I_1$. A similar technique can be applied to reduce $I_n^\prime$ to $I_{-1}^\prime$ and $I_0^\prime=I_0$. 
\end{enumerate}

\begin{idea}
	The irreducible incomplete integrals are those of the first, second, and third kind:
\eq{
I_0 &= \int \frac{1}{\sqrt{(1-x^2)(1-k^2x^2)}}	dx \\
I_0 - k^2 I_1 &= \int \sqrt{\frac{1-k^2 x^2}{1-x^2}} dx \\
I_{-1}^\prime &= \int \frac{1}{(x^2-c^2)\sqrt{(1-x^2)(1-k^2x^2)}}	dx
}
\end{idea}

A lemma useful in the reduction is the following:
\begin{prop}
	Suppose the map sending $e_0, \dots, e_3 \mapsto \pm 1, \pm \frac{1}{k}$ is 
	\eq{
f(z) = \frac{az+b}{cz+d}	
}
Then if we make the substitution $t = \frac{ax+b}{cx+d}$ we have 
\eq{
Q(x) = \prod(x-e_i) = \frac{(1-t^2)(1-k^2t^2)}{k^2(a-ct)^4 \prod (ce_i+d)^{-1}}
}
and 
\eq{
dx = \frac{ad-bc}{(a-ct)^2} dt 
}
\end{prop}
\begin{proof}
The inverse transform is 
	\eq{
x = \frac{dt-b}{a-ct}	
}
so 
\eq{
x-e_i &= \frac{(dt-b) - e_i(a-ct)}{a-ct} \\
&= \frac{(ce_i+d)t-(ae_i+b)}{a-ct} \\
&= \frac{t - f(e_i)}{(a-ct)(ce_i+d)^{-1}}
}
The first part then follows by taking the product and knowing that the $f(e_i)$ are the roots of $(1-x^2)(1-k^2x^2)$, and mathching up the coefficient of $t^4$. For the second part we directly calculate. 
\eq{
dx &= \frac{d }{a-ct}(dt) +  \frac{dt-b}{(a-ct)^2}c(dt) \\
&= \frac{ad-bc}{(a-ct)^2} \, (dt)
}
\end{proof}

\begin{corollary}
	We have
	\eq{
\int \frac{dx}{\sqrt{(x-e_0)(x-e_1)(x-e_2)(x-e_3)}}= \pm \frac{k(ad-bc)}{\prod \sqrt{ce_i + d}} \int \frac{dt}{\sqrt{(1-t^2)(1-k^2t^2)}}	
}
where the $\pm$ fixes the correct choice of sign. 
\end{corollary}

\begin{lemma}
	The map sending $-1,\frac{1}{k},-\frac{1}{k} \mapsto 0,1,\infty$ is 
	\eq{
z \mapsto \frac{2k}{1+k} \frac{1+z}{1+kz}	
} 
and the inverse is 
\eq{
z \mapsto \frac{1}{k}\frac{(1+k)z-2k}{-(1+k)z+2}
}
\end{lemma}
\begin{remark}
	\hl{The above map isn't normalised yet, so fix it}
\end{remark}

\begin{example}
	Consider the integral 
	\eq{
I = \int_0^2 \frac{1}{\sqrt{(2x-x^2)(4x^2+9)}} dx	
}
We start by making $Q$ monic by pulling out the factor of $2$. This gives 
	\eq{
	I = \frac{1}{2}\int_0^2 \frac{1}{\sqrt{(2x-x^2)(x^2+\sfrac{9}{4})}} dx	
}
As such we have $Q(x) = -x(x-2)(x-\sfrac{3i}{2})(x+\sfrac{3i}{2})$. We can  calculate the cross ratio for sending $0 \mapsto C, 2 \mapsto 0, \frac{3i}{2} \mapsto 1, \frac{-3i}{2}\mapsto \infty$:
\eq{
C = \frac{0-2}{0+\sfrac{3i}{2}}\frac{\sfrac{3i}{2}+\sfrac{3i}{2}}{\sfrac{3i}{2}-2} = \frac{-4}{\sfrac{3i}{2}-2} = \frac{2i}{\sfrac{3}{4}+i}
}
This corresponds to a value of $k=\frac{i}{2}$. We want to construct a map sending $0,2,\frac{3i}{2},-\frac{3i}{2} \mapsto 1,-1,\frac{1}{k}, -\frac{1}{k}$ so we consider 
\eq{
z \mapsto \frac{z+b}{cz+d}
}
Imposing the conditions finds $b=d=-(1+c) = -\frac{3}{4}$. We can, without changing our answer, rescale all of these to integers (namely $a=4, b=-3, c=-1, d=-3$) to make the calculations easier. For this transform we find $ad-bc=-15$. Further
\eq{
- 0 -3&= -3 \\
- 2 -3&= - 5 \\
-\frac{3i}{2} -3&= -\pround{3 + \frac{3i}{2}}\\
- \frac{-3i}{2} -3&= -\pround{3 - \frac{3i}{2}}
} 
giving 
\eq{
\prod (ce_i + d)^{-\frac{1}{2}} = \frac{2}{15\sqrt{3}}
}
so after the transform 
\eq{
I = \frac{-15i}{4}\times \frac{2}{15\sqrt{3}} \int_1^{-1} \frac{dt}{\sqrt{-(1-t^2)(1-k^2t^2)}} = \frac{1}{\sqrt{3}}K\pround{\frac{i}{2}}
}
for $k=\frac{i}{2}$. Using 
\eq{
K(ik) &= \frac{1}{\sqrt{1+k^2}} K\pround{\frac{k}{\sqrt{1+k^2}}} \\
\Rightarrow K\pround{\frac{i}{2}} &= \frac{2}{\sqrt{5}} K \pround{\frac{1}{\sqrt{5}}}
}
we get 
\eq{
I = \int_0^2 \frac{1}{\sqrt{(2x-x^2)(4x^2+9)}} dx = \frac{2}{\sqrt{15}}K\pround{\frac{1}{\sqrt{5}}}
}
\end{example}

%%%%%%%%%%%%%%%%%%%%%%%%%%%%%%%%%%%%%%%%%%%%%%%%%%%%%%%%
\subsection{Connection to Elliptic Functions}

Recall that when we defined the incomplete elliptic integrals,

Because of the differential equation for $\wp$ we have the following result:

\begin{prop}
We have 
\eq{
z-z_0 = \frac{1}{2} \int_{\wp(z_0)}^{\wp(z)} \frac{dt}{\sqrt{(t-e_1)(t-e_2)(t-e_3)}} \quad \mod L
}
where the each value of the integral depends on the contour chosen. 
\end{prop}

\begin{example}
	Choosing the correct sign of the radicals, we have 
	\eq{
\omega_1 &= \int_{\infty}^{e_1} \frac{dt}{\sqrt{(t-e_1)(t-e_2)(t-e_3)}} \\
 \omega_2 &= \int_{e_1}^{e_2} \frac{dt}{\sqrt{(t-e_1)(t-e_2)(t-e_3)}}
}
\end{example}

We can also use the Weierstrass function to solve elliptic integrals.

\begin{example}
	Consider the integral 
	\eq{
I(x) = \int_{\infty}^x \frac{dt}{\sqrt{t^4 + 6at^2 + b^2}}	
}
for $a,b \in \mbb{R}$. Make the substitution $s=t^2$ to get 
\eq{
I(x) = \frac{1}{2} \int_{\infty}^{x^2} \frac{ds}{\sqrt{s^3+ 6as^2 + b^2 s }} 
}
To put this bottom cubic in Weierstrass form we need to remove the $s^2$ term, which can be achieved by making the translation $u = s+2a$. Then 
\eq{
I(x) &= \frac{1}{2}\int_{\infty}^{x^2+2a} \frac{du}{\sqrt{u^3 - (12a^2-b^2)u - 2a(b^2-8a^2)}} \\
&= \int_{\infty}^{x^2+2a} \frac{du}{\sqrt{4u^3 - 4(12a^2-b^2)u - 8a(b^2-8a^2)}}
}
This is in Weierstrass form with $g_2 = 4(12a^2-b^2), \, g_3 = 8a(b^2-8a^2)$. The final result is 
\eq{
I(x) =  \wp^{-1}(x^2+2a \, | \, 4(12a^2-b^2), \, 8a(b^2-8a^2))
} 

\end{example}

%%%%%%%%%%%%%%%%%%%%%%%%%%%%%%%%%%%%%%%%%%%%%%%%%%%%%%%%
%%%%%%%%%%%%%%%%%%%%%%%%%%%%%%%%%%%%%%%%%%%%%%%%%%%%%%%%
%%%%%%%%%%%%%%%%%%%%%%%%%%%%%%%%%%%%%%%%%%%%%%%%%%%%%%%%
%%%%%%%%%%%%%%%%%%%%%%%%%%%%%%%%%%%%%%%%%%%%%%%%%%%%%%%%
\part{Curves}

%%%%%%%%%%%%%%%%%%%%%%%%%%%%%%%%%%%%%%%%%%%%%%%%%%%%%%%%
%%%%%%%%%%%%%%%%%%%%%%%%%%%%%%%%%%%%%%%%%%%%%%%%%%%%%%%%
\section{Abel's Inversion Theorem}

We start by noting to following connection, a complex torus $X = \faktor{\mbb{C}}{L}$ with $e_1, e_2. e_3$ given, has the behaviour 
\eq{
[\wp^\prime]^2 = 4(\wp-e_1)(\wp-e_2)(\wp-e_3)
}
in it's function field. 
	
%%%%%%%%%%%%%%%%%%%%%%%%%%%%%%%%%%%%%%%%%%%%%%%%%%%%%%%%
%%%%%%%%%%%%%%%%%%%%%%%%%%%%%%%%%%%%%%%%%%%%%%%%%%%%%%%%
\bibliographystyle{../bib/custom-bib-style}
\bibliography{../bib/library,../bib/manual}

\end{document}