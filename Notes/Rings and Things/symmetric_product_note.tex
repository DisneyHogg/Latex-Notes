\documentclass{article}

\usepackage{../../header}
%%%%%%%%%%%%%%%%%%%%%%%%%%%%%%%%%%%%%%%%%%%%%%%%%%%%%%%%
%Preamble

\title{}
\author{Linden Disney-Hogg}
\date{}

%%%%%%%%%%%%%%%%%%%%%%%%%%%%%%%%%%%%%%%%%%%%%%%%%%%%%%%%
%%%%%%%%%%%%%%%%%%%%%%%%%%%%%%%%%%%%%%%%%%%%%%%%%%%%%%%%
\begin{document}

\maketitle
\tableofcontents

%%%%%%%%%%%%%%%%%%%%%%%%%%%%%%%%%%%%%%%%%%%%%%%%%%%%%%%%
%%%%%%%%%%%%%%%%%%%%%%%%%%%%%%%%%%%%%%%%%%%%%%%%%%%%%%%%
\section{A note on symmetric products}
In Atiyah-Sutcliffe-2002, Equation 3.3, there is a typo. Here we shall explain the correct version, and clear up a misconception. \\
Let's start with two identifications. We consider $\mathbb{C}^{n+1}$ to be homogenous polynomials in $\mathbb{C}[x,y]$ of degree $n$, which we do via the map
\[
(a_0, a_1, \dots, a_n) \mapsto a_0 x^n + a_1 x^{n-1}y + \dots , a_n y^n
\]
This means we can consider $\pround{\mathbb{C}^2}^d$ to be list\footnote{Note I say lists here, as the ordering currently matters.} of the form $[\alpha_i x + \beta_i y \, | \, i=1, \dots, d]$. One therefore might expect there to be some correspondence 
\begin{align*}
	(\mathbb{C}^2)^n &\to \mathbb{C}^n \, ,  \\
[\alpha_i x + \beta_i y] &\mapsto \prod_i (\alpha_i x + \beta_i y) \, .
\end{align*} 
Such a correspondence (in the sense of Ward and Wells) does indeed exist, and we want to see what conditions we can impose to make this into a bijection. The map is well-defined and onto, but will fail to be injective because we can permute the $(\alpha_i, \beta_i)$, and also scale each pair by $\lambda_i$ s.t $\prod_i \lambda_i =1$. Let's state a somewhat obvious result for now, which will be important later. 
\begin{lemma}
	Given manifolds $X, \, Y$, we have 
	\[
	\dim (X \times Y) = \dim (X) + \dim (Y) \, ,
	\]
	where $\times$ is the (cartesian) product in our category of manifolds. 
\end{lemma}
\begin{lemma}
	A vector space over field $\mathbb{F}$ is also a manifold over $\mathbb{F}$ and the two concepts of dimension agree.  
\end{lemma}
Counting dimensions, we see the LHS has $2n$, while the RHS has $n+1$. We will now see two ways to rectify this. 
%%%%%%%%%%%%%%%%%%%%%%%%%%%%%%%%%%%%%%%%%%%%%%%%%%%%%%%%
\subsection{Projectivise then symmetric product}
In
Recall now a
\begin{definition}
	The \bam{$n$-fold symmetric product} of a topological space is 
	\[
	\operatorname{Sym}^n(X) := \faktor{X^n}{S_n} \, ,
	\]
	the the group action of $S_n$ is the obvious action of permuting the entries in the $n$-fold product.  
\end{definition} 
We can view elements of $\operatorname{Sym}^n(\mathbb{C}^2)$ as sets of the form $\{\alpha_i x + \beta_i y \, | \, i=1, \dots, d\}$, and so the correspondence 
\[
	\operatorname{Sym}^n(\mathbb{C}^2) \to \mathbb{C}^{n+1}
\]
does not have a permutation over-counting issue. To fix the scaling we need to first projectivise each $\mathbb{C}^{k} \to \mathbb{P}^{k-1}$ first. This gives the correspondence 
\[
	\operatorname{Sym}^n(\mathbb{P}^1) \to \mathbb{P}^n \, ,
\]
and counting dimensions give $n$ on both the LHS and RHS, that is we have 
\begin{equation}
	\operatorname{Sym}^{n}(\mathbb{P}^1) \cong \mathbb{P}^n \, , \tag{A}
\end{equation}
where the notion of equality is homeomorphism of topological spaces. 
%%%%%%%%%%%%%%%%%%%%%%%%%%%%%%%%%%%%%%%%%%%%%%%%%%%%%%%%
\subsection{The symmetric tensor product}
Now $\mathbb{C}^n$ is not just a topological space, but also has an algebraic structure as a vector space over $\mathbb{C}$. We then have a different product, which we recall now.
\begin{definition}
	Given vector spaces $V, W$ we define the tensor product 
	\[
	V \otimes W := \faktor{F(V \times W)}{R}
	\]
	where $\times$ is the cartesian product, $F(V \times W)$ is the free vector space with $V \times W$ as a basis and $R$ is the subspace of $V \times W$ spanned by the relations
	\begin{itemize}
		\item $(v_1+v_2, w)-(v_1, w)-(v_2, w)=0$,
		\item $(v, w_1+w_2)-(v,w_1)-(v,w_2)=0$,
		\item $(\lambda v, w) = \lambda(v,w) = (v, \lambda w)$.
	\end{itemize}
\end{definition}
We now give a counterpart to a previous lemma.
\begin{lemma}
	$\dim (V \otimes W) = \dim (V) \times \dim(W)$. 
\end{lemma}
We can see that the ability to pass scalars through the tensor product will fix our issue of overcounting on the scaling. We thus need only take care of the symmetry, which we do with the following definition.
\begin{definition}
	We can likewise define the $n$-fold symmetric tensor product to be 
	\[
	\operatorname{Sym}^{\otimes n}(V) := \faktor{V^{\otimes n}}{S_n} \, .
	\]
\end{definition}
Note the quotient here is really quotienting by a subspace generated by elements of the form $v \otimes w - w \otimes v$, and as such stands to reduce the dimension of the vector space. This occurs as per the following lemma.
\begin{lemma}
	Let $V$ be a vector space of dimension $n$. Then 
	\[
	\dim \operatorname{Sym}^{\otimes d}(V) = \begin{pmatrix}
	n + d-1 \\ d
	\end{pmatrix} \, .
	\]
\end{lemma}
\begin{proof}
	Given a basis $\{e_i \, | i=1, \dots, n\}$ for $V$, we have a basis for $\operatorname{Sym}^{\otimes d}(V)$ given by 
	\[
	\{ e_{i_1} \otimes \dots \otimes e_{i_d} \, | \, 1 \leq i_1 \leq \dots \leq i_d \leq n\} \, .
	\]
	A standard combinatorics argument gives the answer. 
\end{proof}
\begin{example}
	Applying the above we get 
	\[
	\dim \operatorname{Sym}^{\otimes n}(\mathbb{C}^2) = \begin{pmatrix}
	n + 1 \\ n
	\end{pmatrix} = n+1 \, .
	\]
	The basis is given by 
	\[
	\underbrace{e_1 \otimes \dots \otimes e_1}_{\times k} \otimes  \underbrace{e_2 \otimes \dots \otimes e_2}_{\times n-k} \, . 
	\]
	This gives us an important result
	\begin{equation}
	\operatorname{Sym}^{\otimes n}(\mathbb{C}^2) \cong \mathbb{C}^{n+1} \, , \tag{B}
	\end{equation}
	where the notion of equality is isomorphism of vector spaces. 
\end{example}

%%%%%%%%%%%%%%%%%%%%%%%%%%%%%%%%%%%%%%%%%%%%%%%%%%%%%%%%
\subsection{What is happening here}
At this point one might start to be worried about how the dimensions could have possibly worked out. In one case the product added dimensions, and taking the quotient by a group action didn't reduce the dimension, and in one case the product multiplied the dimension, then taking the quotient reduced the dimension. \\
To understand more, realise the direct product in the category of vector spaces is the standard product which exists for any category, in this case isomorphic to the direct sum for finite products, whereas the tensor product is an additional operation, in this case imparting some information about multilinearity. Both direct sum and tensor product make the category of finite dimensional vector spaces into a monoidal category, which decategorifies to $\mathbb{N}$, but $(\text{FinVect}, \oplus)$ decategorifies into $(\mathbb{N}, +)$, while $(\text{FinVect}, \otimes)$ decategorifies into $(\mathbb{N}, \times)$. \\
To understand then why the quotient by the symmetric group has the effect on dimension it does, note that when we quotient by the group action on topological spaces, we are identifying points, whereas when it acts on vector spaces we are identifying basis vectors. 

%%%%%%%%%%%%%%%%%%%%%%%%%%%%%%%%%%%%%%%%%%%%%%%%%%%%%%%%
%%%%%%%%%%%%%%%%%%%%%%%%%%%%%%%%%%%%%%%%%%%%%%%%%%%%%%%%
\bibliographystyle{../bib/custom-bib-style}
\bibliography{../bib/jabref_library.bib}

\end{document}
