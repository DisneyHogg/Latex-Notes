\documentclass{article}

\usepackage{../../header}
%%%%%%%%%%%%%%%%%%%%%%%%%%%%%%%%%%%%%%%%%%%%%%%%%%%%%%%%
%Preamble

\title{Affine Toda}
\author{Linden Disney-Hogg}
\date{September 2020}

%%%%%%%%%%%%%%%%%%%%%%%%%%%%%%%%%%%%%%%%%%%%%%%%%%%%%%%%
%%%%%%%%%%%%%%%%%%%%%%%%%%%%%%%%%%%%%%%%%%%%%%%%%%%%%%%%
\begin{document}

\maketitle
\tableofcontents

%%%%%%%%%%%%%%%%%%%%%%%%%%%%%%%%%%%%%%%%%%%%%%%%%%%%%%%%
%%%%%%%%%%%%%%%%%%%%%%%%%%%%%%%%%%%%%%%%%%%%%%%%%%%%%%%%
\section{Introduction}
These will be a set of notes dedicated to a project looking at the affine toda lattice, but in situ we will cover some theory from Lie algebras and representations. See my notes on Kac-Moody algebras and Symmetries, Fields, and Particles for additional background which I will omit here as it is covered there. 


  \section{Lie Algebra Conventions}
Let $\mathfrak{g}$ be a simple Lie algebra of rank $r$ and $\mathfrak{h}\subset \mathfrak{g}$ 
a fixed Cartan subalgebra with a  inner product $(\  ,  \ ):=(\  ,  \ )_{\mathfrak{h}\sp\ast}$. 
Let $\Phi$ denote the set of roots for the pair $(\mathfrak{g},\mathfrak{h})$ and $W$ the associated Weyl group. By averaging we may always take  $(\  ,  \ )$ to be Weyl-invariant. We begin with 
\begin{enumerate}[(i)]
	\item the linearly independent set $\Delta:=\{\alpha_1,\ldots,\alpha_r\}\subset\Phi\subset \mathfrak{h}\sp\ast,$ the simple roots. To each $\alpha\in\Phi$ set
	$$\epsilon_\alpha:=\frac2{(\alpha,\alpha)},\quad \alpha\sp\vee :=\epsilon_\alpha \alpha:=\frac{2\alpha}{(\alpha,\alpha)}.
	$$
	Here $\alpha\sp\vee\in \mathfrak{h}\sp\ast$ are the \textbf{coroots} (or \textbf{dual} roots)
	and $\Phi\sp\vee:=\{\alpha\sp\vee \,|\,\alpha\in\Phi\}$\footnote{Caution:
		Kac's notation has $\alpha\sp\vee\in \mathfrak{h} $}. We write $\epsilon_i:={2}/{(\alpha_i ,\alpha_i)}$
	for $\alpha_i\in\Delta$.
	\item The Cartan matrix is $A:=(a_{ij})$ with $ a_{ij}:=(\alpha_i\sp\vee,\alpha_j)$. Then 
	$A=DB$ where $D=\diag(\epsilon_1,\ldots,\epsilon_n)$ and $B:=(b_{ij})$ , $b_{ij}=(\alpha_i ,\alpha_j)$ is symmetric; $A$ is symmetrizable. Then
	$$(\alpha_i\sp\vee,\alpha_j\sp\vee)=\epsilon_i(\alpha_i ,\alpha_j)\epsilon_j=\epsilon_i \,\alpha_i(\alpha_j\sp\vee).
	$$
	The choice of $\epsilon_\alpha$ is so as to make the Cartan matrix have
	two's along the diagonal, 
	
	
	\item Let $\{H_a\}$ ($a=1,\ldots,r$) be a basis of ${\mathfrak{h}}$.
	The Cartan-Weyl basis $\{H_a\}$ and $\{E_\alpha\}$, $\alpha\in\Phi$ satisfies
	$$[H_a,H_b]=0,\quad[H_a,E_\alpha]=\alpha_a\, E_\alpha, \quad \alpha_a:=\alpha(H_a).
	$$
	The Jacobi identity then yields  for $\alpha,\beta\in\Phi$ that
	$$[H_a,[E_\alpha,E_\beta]]=(\alpha+\beta)_a\,[E_\alpha,E_\beta]$$
	and so
	$$[E_\alpha,E_\beta]=
	\begin{cases} c_{\alpha,\beta}E_{\alpha+\beta}&\text{if }\alpha+\beta\in\Phi,\\
	0&\text{if }\alpha+\beta\ne0\text{ and }\alpha+\beta\not\in\Phi.
	\end{cases}
	$$
	Finally, using the fact that the $\text{centraliser}_\mathfrak{g}(\mathfrak{h})=\mathfrak{h}$ we
	see that $[E_\alpha,E_{-\alpha}]\in \mathfrak{h}$.
	
	\item Denote the Killing form by 
	\begin{equation}
	\kappa(x,y):=\tr \ad_x\circ \ad_y, \qquad x,y\in\mathfrak{g}.
	\end{equation}
	Then
	$$ \kappa([x,y],z)=\kappa(x,[y,z]).
	$$
	The non-degeneracy of the Killing form means we get an
	isomorphism $\nu:
	{\mathfrak{h}}\rightarrow{\mathfrak{h}}\sp\ast$ such that
	$\kappa(h_1 ,  h_2 )_{\mathfrak{h}}=\nu(h_1)(h_2)$. For each $\alpha\in\Phi$ define $t_\alpha\in \mathfrak{h}$ by $\nu(t_\alpha)=\alpha$.  Thus $\alpha(t_\alpha)=\kappa(t_\alpha,t_\alpha)$.
	Then for all $h\in {\mathfrak{h}}$
	\begin{align*}
	\kappa(h, [E_\alpha,E_{-\alpha}])&=\kappa([h, E_\alpha],E_{-\alpha}])=\alpha(h)
	\kappa(E_\alpha,E_{-\alpha} )=\kappa(t_\alpha,h )\kappa(E_\alpha,E_{-\alpha}  )\\
	&=\kappa(\kappa(E_\alpha,E_{-\alpha}  )\, t_\alpha,h  ).
	\end{align*}
	and the non-degeneracy of the Killing form  now yields that
	$$ [E_\alpha,E_{-\alpha}]=\kappa(E_\alpha,E_{-\alpha}  )\, t_\alpha.$$
	
	\item Upon noting that
	\begin{align*}
	\ad_{H_a}\circ\ad_{H_b}(h)&=0
	\\
	\ad_{H_a}\circ\ad_{H_b}(E_\alpha)&=\alpha_a \alpha_b\,E_\alpha
	\end{align*}
	we find
	$$
	\kappa(H_a,H_b)=\sum_{\alpha\in \Phi}\alpha_a \alpha_b.
	$$
	
	\item The Weyl group acts irreducibly on the vector space $\mathfrak{h}\sp\ast$. If we write the $W$-invariant metric as $(\alpha,\beta)=\alpha_a g\sp{ab}\beta_b$ then
	$$\sum_{w\in W} (w\alpha)_a (w\alpha)_b =\frac{(\alpha,\alpha)}{r}\,|\mc{O}(\alpha)|\, g_{ab}.$$
	Now a root system $\Phi$ consists of at most root vectors of two lengths two (long $L$ and short $S$), and those vectors of the same length form a single orbit. Then
	$$\sum_{\alpha\in \Phi}\alpha_a \alpha_b
	=\left( (\alpha_L,\alpha_L)\,|\mc{O}(\alpha_L)|+(\alpha_S,\alpha_S)\,|\mc{O}(\alpha_S)|\right) \, g_{ab}
	=2g\,\frac{ (\alpha_L,\alpha_L)}2\, g_{ab}
	.
	$$
	Here $g$ is the \textbf{dual Coxeter} number. Therefore
	$$
	\kappa(H_a,H_b)=2g\,\frac{ (\alpha_L,\alpha_L)}2\, g_{ab}.
	$$
	
	\item Let us set $c:=2g\, (\alpha_L,\alpha_L)/2$ so that $\kappa_{ab}:=\kappa(H_a,H_b)=c\, g_{ab}$. 
	We wish to express
	$t_\alpha$ in terms of the basis $\{H_a\}$. Now 
	$$\kappa(t_\alpha,H_a)=\nu(t_\alpha)(H_a)=\alpha(H_a)=\alpha_a.$$
	If $t_\alpha=x\sp{b}H_b$ then $x\sp{b}\kappa_{ba}=\alpha_a$ and so $x\sp{b}=\alpha_a g\sp{ab}/c=
	\alpha\sp{b}/c$ and
	$$t_\alpha=\frac1{c}\alpha\sp{a}H_a=\frac1{c}\alpha\cdot H.$$
	Note that
	$$\alpha(t_\alpha)=\kappa(t_\alpha,t_\alpha)=\frac{\alpha\sp{a}}{c}\,\kappa(H_a,H_b)\,\frac{\alpha\sp{b}}{c}
	=\frac{\alpha\sp{a}}{c}\,c\, g_{ab}\,\frac{\alpha\sp{b}}{c}=\frac{(\alpha,\alpha)}{c}.
	$$
	
	\item Set
	$$H_\alpha:=\frac{2\,t_\alpha}{\kappa(t_\alpha,t_\alpha)}=\frac{2\,\alpha\cdot H}{(\alpha,\alpha)}
	=\alpha\sp\vee\cdot H.
	$$
	Upon noting that $[t_\alpha, E_\alpha]=\alpha(t_\alpha)E_\alpha=(\alpha,\alpha)E_\alpha/{c}$
	then for all $\alpha\in\Phi$,
	$$[H_\alpha, E_\alpha]=2\,E_\alpha.
	$$
	Now
	$$ [E_\alpha,E_{-\alpha}]=\kappa(E_\alpha,E_{-\alpha})\,t_\alpha=
	\left[\frac12 \kappa(E_\alpha,E_{-\alpha}) \kappa(t_\alpha,t_\alpha) \right] H_\alpha.
	$$
	Setting
	$$E_\alpha\sp{Ch}:=E_\alpha/\sqrt{\frac12 \kappa(E_\alpha,E_{-\alpha}) \kappa(t_\alpha,t_\alpha) }$$
	we then have for  all $\alpha\in\Phi$ the standard $sl_2$ relations
	$$[H_\alpha, E_\alpha\sp{Ch}]=2\,E_\alpha\sp{Ch},\quad [E_\alpha\sp{Ch},E_{-\alpha}\sp{Ch}]=H_\alpha.
	$$
	Further
	$$[H_\alpha, E_\beta\sp{Ch}]=\epsilon_\alpha \alpha\sp{a} \beta(H_a) E_\beta\sp{Ch}=
	(\alpha\sp\vee,\beta)\, E_\beta\sp{Ch}
	$$
	and
	$$\kappa(H_\alpha,H_\beta)=c\,(\alpha\sp\vee,\beta\sp\vee),\quad
	\kappa(E_\alpha\sp{Ch},E_{-\alpha}\sp{Ch})=c\,\epsilon_\alpha.
	$$
	
	
	\item The Chevalley basis consists of
	$\{H_\alpha\}$ for $\alpha\in\Delta$ and $\{E_\beta\sp{Ch}\}_{\beta\in\Phi}$, 
	where
	\begin{align*} 
	[H_\alpha, E_\beta\sp{Ch}]&=(\alpha\sp\vee,\beta)\, E_\beta\sp{Ch},\\
	[E_\alpha\sp{Ch},E_\beta\sp{Ch}]&=
	\begin{cases} H_\alpha
	&\text{if }\alpha+\beta=0,\\
	N_{\alpha,\beta}E_{\alpha+\beta}&\text{if }\alpha+\beta\in\Phi,\\
	0&\text{if }\alpha+\beta\ne0\text{ and }\alpha+\beta\not\in\Phi.
	\end{cases}
	\end{align*}
	with
	$$\kappa(H_\alpha,H_\beta)=c\,(\alpha\sp\vee,\beta\sp\vee),\quad
	\kappa(E_\alpha\sp{Ch},E_{-\alpha}\sp{Ch})=c\,\epsilon_\alpha, \quad
	c=2g\, \frac{(\alpha_L,\alpha_L)}{2}.
	$$
	\item There is a unique maximal root, which we denote as $\Theta=\sum_{\alpha\in\Delta} {n_\alpha}\, \alpha$ be the highest root. Set 
	$\bar\Delta=\Delta\cup\{-\Theta\}$
\end{enumerate}
%%%%%%%%%%%%%%%%%%%%%%%%%%%%%%%%%%%%%%%%%%%%%%%%%%%%%%%%
%%%%%%%%%%%%%%%%%%%%%%%%%%%%%%%%%%%%%%%%%%%%%%%%%%%%%%%%
\section{Background Theory}
To see monopoles reading, look at my monopole notes. This affine Toda may eventually be folded in. 
%%%%%%%%%%%%%%%%%%%%%%%%%%%%%%%%%%%%%%%%%%%%%%%%%%%%%%%%
\subsection{Lie Algebras and Representation Theory}
We start with a recap of Chapters II and III of \cite{Humphreys1978}. Denote the base of simple roots as $\Delta$. 

\begin{prop}
	There exists a unique root of highest weight $\theta = \sum_{\alpha \in \Delta} n_\alpha \alpha \in \mf{h}^\ast$. 
\end{prop}

\begin{prop}
	Let $A$ be the cartan matrix corresponding to $\mf{g}$ of finite type, rank $n$, and let $h_\theta = \sum_i n_i h_i\in \mf{h}$ be the element corresponding to $\theta$ under the natural iso $\mf{h} \cong \mf{h}^\ast$. Define $\hat{A}$ by 
	\eq{
	\hat{A}_{ij} &= A_{ij}, \, 1 \leq i,j \leq n \\ 
	\hat{A}_{00} &= 2 \\
	\hat{A}_{i0} &= -\sum_j m_j A_{ij} \\
	\hat{A}_{0j} &= -\sum_i n_i A_{ij} 
}
Then $\hat{A}$ is an affine generalised Cartan matrix corresponding to an \bam{untwisted affine Dynkin diagram}.  
\end{prop}

\begin{prop}
	The Lie algebra corresponding to $\hat{A}$ is isomorphic to the affine Kac-Moody Lie algebra $\mc{L}\mf{g} \oplus \mbb{C}c \oplus \mbb{C}d$
\end{prop}


%%%%%%%%%%%%%%%%%%%%%%%%%%%%%%%%%%%%%%%%%%%%%%%%%%%%%%%%
%%%%%%%%%%%%%%%%%%%%%%%%%%%%%%%%%%%%%%%%%%%%%%%%%%%%%%%%
\section{Affine Toda}
 We start by introducing affine Toda from a field theory perspective, following \cite{Braden1990}:
 
%%%%%%%%%%%%%%%%%%%%%%%%%%%%%%%%%%%%%%%%%%%%%%%%%%%%%%
\subsection{Independent Definition}
 
\begin{definition}
	Let $\mf{g}$ be a rank-$r$ Lie algebra with simple roots $\alpha_i$, taking a particular realisation of these as vectors in $\mbb{R}^r$. The \bam{Toda field theory} is that with $\mbb{R}^r$-valued field $\bm{\Phi} = (\phi^a)$ and Lagrangian 
	\eq{
\mc{L} = \frac{1}{2} \del_\mu \phi^a \del^\mu \phi^a - \frac{\lambda}{\beta^2} \sum_{i=1}^r e^{\beta \alpha_i \cdot \bm{\Phi}}	
} 
for parameters $\lambda,\beta$.
\end{definition}

\begin{prop}
	The corresponding classical equations of motion are 
	\eq{
\del^2 \phi_j = -\frac{\lambda}{\beta} \sum_{i=1}^r C_{ji} e^{\beta \phi_i}	
}
where $\phi_j = \alpha_j \cdot \bm{\Phi}$ and 
\eq{
C_{ij} = \alpha_i \cdot \alpha_j
}
\end{prop}
\begin{proof}
	The e.o.m are 
	\eq{
\pd[\mc{L}]{\phi^a} &= \del_\mu \pd[\mc{L}]{\del_\mu \phi^a} \\
\Rightarrow -\frac{\lambda}{\beta} \sum_{i=1}^r \pround{\alpha_i}^a e^{\beta \phi_i} &= \del^2 \phi^a 
}
and the result follow from contracting with $\alpha_j$.
\end{proof}

\begin{remark}
	If we shift $\phi_i \mapsto \phi_i + \frac{1}{\beta}\log \pround{\frac{2}{\alpha_i^2}}$ the matrix $C$ is replaced with 
	\eq{
A_{ij} = \frac{2\alpha_i \cdot \alpha_j}{\alpha_j^2}	
}
which we recognise to be the Cartan matrix. 
\end{remark}

This field theory does not have a unique minimum for us to consider as the classical vacuum, so we will want to deform it in some way. The following result motivates how do this deformation:

\begin{prop}
	$1+1$-dimensional Toda field theory has a zero-curvature representation 
\end{prop}
\begin{proof}
	We follow \cite{Olive1983}. Define light-cone coordinates
	\eq{
u &= \frac{1}{2}(x+t) \\
v &= \frac{1}{2}(x-t)	
}
s.t. 
\eq{
\del_u \del_v = -\del_t^2 + \del_x^2 = - \del_\mu \del^\mu
}
and a gauge potential with 
\eq{
A_u &= \sum_{i=1}^r \pround{\frac{1}{2}}
}
\hl{finish this off...}
\end{proof}

This is thus integrable, and so when we want to generalise this system, we look for integrable deformations. 

\begin{definition}
	The field theory obtained by perturbing Toda by 
	\eq{
\delta V(\bm{\Phi}) = \frac{\eps\lambda}{\beta^2} e^{\beta \alpha_0 \cdot \bm{\Phi}}	
}
s.t. $\pbrace{\alpha_0, \alpha_j}$ are roots of an affine Lie algebra is called \bam{affine Toda field theory}. 
\end{definition}

Affine Toda has a minimum $\bm{\Phi}^{(0)}$ satisfying 
\eq{
\sum_i \alpha_i e^{\beta \alpha_i \cdot \bm{\Phi}^{(0)}} = -\eps \alpha_0 e^{\beta \alpha_0 \cdot \bm{\Phi}^{(0)}}
}
If we centre around this by letting $\bm{\Phi} = \bm{\Phi}^{(0)} + \bm{\phi}$ we have 
\eq{
V(\bm{\phi}) &= \frac{\eps\lambda}{\beta^2} e^{\beta \alpha_0 \cdot \bm{\Phi}^{(0)}}\psquare{e^{\beta \alpha_0 \cdot \bm{\phi}}-\sum_{i,j} e^{\beta \alpha_i \cdot \bm{\phi}}\pround{C^{-1}}_{ij}\alpha_j \cdot \alpha_0} \\
&= \frac{m^2}{\beta^2} \sum_{i=0}^r n_i e^{\beta \alpha_i \cdot \bm{\phi}}
}
where we have let $m^2 = \eps \lambda e^{\beta \alpha_0 \cdot \bm{\Phi}^{(0)}}$ and written $\alpha_0 = \sum_{i=1}^r n_i \alpha_i, \, n_0=1$. The simplest deformation is to take $\alpha_0 = -\Theta$, the maximal root.  

\begin{example}
	Consider the Lie algebra $A_n$. We may find a realisation of the $n$ simple where $\alpha_i = e_i - e_{i+1}$, where $e_i$ are the canonical basis vectors of $\mbb{R}^n$ and we take $e_{n+1}=e_1$. Moreover, the maximal root has all $n_i=1, \, 1\leq i \leq n$. 
\end{example}
%%%%%%%%%%%%%%%%%%%%%%%%%%%%%%%%%%%%%%%%%%%%%%%%%%%
\subsection{Braden Approach}
We will now obtain the affine Toda Field theory through a different lens. \\
Let $\mathfrak{g}$ be a compact semisimple Lie algebra of rank $r$ with a fixed Cartan subalgebra $\mathfrak{h}$.
Let $\{X_\mu\}=\{H_i, E_\alpha\}$ be a Cartan-Weyl basis where $\{H_i\}$ is a basis of $\mathfrak{h}$
and $\{E_\alpha\}$ the set of step operators (labelled by the root system $\Phi$ of $\mathfrak{g}$) and 
$$[H_i,E_\alpha]=\alpha_i\, E_\alpha,\quad [E_\alpha,E_{-\alpha}]=\alpha\cdot H,
\quad [E_\alpha, E_\beta]=N_{\alpha,\beta}\, E_{\alpha+\beta} \quad\text{if }\alpha+\beta\in \Phi.
$$
Recalling we have the maximal root $\Theta = \sum_{\alpha \in \Delta} n_\alpha \alpha$, and extend $\bar{\Delta} = \Delta \cup \pbrace{-\Theta}, \, n_{-\Theta} = 1$ consider 
$$E=\sum_{\alpha\in \bar\Delta} \sqrt{n_\alpha}\, E_\alpha,\quad
E\sp\dagger=\sum_{\alpha\in \bar\Delta} \sqrt{n_\alpha}\, E_{-\alpha}.
$$
\begin{lemma}
	$\comm[E]{E^\dagger}=0$.
\end{lemma}
\begin{proof}
$$[E,E\sp\dagger]=\sum_{\alpha\in\Delta}  {n_\alpha}\,[E_\alpha, E_{-\alpha}]+[ E_{-\Theta},E_\Theta]
=0$$
\end{proof}
Consider a Lagrangian density
$$\mathcal{L}=  \frac12 \kappa \left( \partial_\mu\phi, \partial\sp\mu\phi\right) -\kappa \left( e^{b\phi}E  e^{-b\phi}, E\sp\dagger \right)
$$
for $\phi = \phi^i H_i$. 
\begin{prop}
	The corresponding field equations are 
	$$\partial_\mu\partial\sp\mu\phi +b\,[e^{b\phi}E  e^{-b\phi}, E\sp\dagger]=0.$$
	These are the affine Toda field equations if we can make the normalisation $\kappa(E_\alpha,E_{-\alpha}) = \frac{m^2}{\lambda^2}$ . 
\end{prop}
\begin{proof}
	We first expand 
	\eq{
\kappa(\del_\mu \phi,\del^\mu \phi) = \del_\mu \phi^i \del^\mu \phi^j \kappa(H^i,H^j)	
}
As $\mf{g}$ is compact and semi-simple we can have chosen the basis of the Cartan subalgebra s.t. $\kappa(H^i,H^j) = -\kappa \delta^{ij}$, and so we get the $\del^\mu \del_\mu \phi$ from the kinetic part of the equations of motion. Then 
\eq{
e^{b\phi} E e^{-b\phi} &= \exp(\ad_{b\phi})E = \sum_{\alpha \in \bar{\Delta}} \sqrt{n_\alpha} e^{b\alpha(\phi)} E_\alpha
}
giving 
\eq{
\kappa \left( e^{b\phi}E  e^{-b\phi}, E\sp\dagger \right) &= \sum_{\alpha,\beta \in \bar{\Delta}} \sqrt{n_\alpha n_\beta} e^{b\alpha(\phi)} \kappa(E_\alpha,E_{-\beta}) \\
&= \sum_{\alpha \in \bar{\Delta}} n_\alpha e^{b\alpha(\phi)} \kappa(E_\alpha,E_{-\alpha}) \quad \text{as }\alpha-\beta \notin \Phi \text{ for }\alpha,\beta \in \bar{\Delta} 
}
Taking the differential wrt to $\phi$ we find that the EL equations are 
\eq{
-\del_\mu \del^\mu \phi &= - b\sum_{\alpha \in \bar{\Delta}} n_\alpha (\alpha \cdot H)e^{b\alpha(\phi)} \kappa(E_\alpha,E_{-\alpha}) \\
&= b \sum_{\alpha \in \bar{\Delta}} n_\alpha t_\alpha e^{b\alpha(\phi)} \kappa(E_\alpha,E_{-\alpha}) \\
&=  b \sum_{\alpha \in \bar{\Delta}} n_\alpha  e^{b\alpha(\phi)} \comm[E_\alpha]{E_{-\alpha}} = b \comm[e^{b\phi}E e^{-b\phi}]{E^\dagger} 
}
\end{proof}

\begin{remark}
	We can rewrite this Lagrangian as
	\eq{
\mc{L} = \frac{1}{2}\tr\psquare{\ad(\del_\mu \phi) \ad(\del^\mu \phi)} -\tr\psquare{ \ad(e^{b\phi}Ee^{-b\phi})\ad(E^\dagger)}	
}
by definition of the Killing form, and so we often make the adjoint rep implicit in this equation to just write 
\eq{
\mc{L} = \tr \psquare{\frac{1}{2} \del_\mu \phi \del^\mu \phi - e^{b\phi}E e^{-b\phi}E^\dagger}
}
\end{remark}

Given the following remark, to make this notation more sensible we show the following results:

\begin{lemma}
If $A,B$ are matrices s.t. $A^\dagger=A$ and $\comm[A]{B} \lambda B$, then $\comm[A]{B^\dagger} = - \bar{\lambda} B^\dagger$. 	
\end{lemma}

\begin{lemma}
	Wrt to the Cartan-Weyl basis, $\ad(H)^\dagger = \ad(H), \, \ad(E_\alpha)^\dagger = \ad(E_{-\alpha})$. 
\end{lemma}
\begin{proof}
	We first assume we have chosen our Lie algebra to a vector space over $\mbb{R}$. Then, we know that the elements of the Cartan subalgebra are diagonalised wrt the CW basis, and their diagonal must be real, so we instantly have that $\ad(H) = \ad(H)^T = \ad(H)^\dagger$. \\
	Now we apply the previous lemma as 
	\eq{
\comm[\ad(H)]{\ad(E_\alpha)} = \ad\pround{\comm[H]{E_\alpha}} = \ad\pround{\alpha(H)E_\alpha} =\alpha(H) \ad(E_\alpha)	
}
to see that $\ad(E_\alpha)^\dagger$ is in the same eigenspace of $\ad(\ad(H))$ as $\ad(E_{-\alpha})$. Whether it is then possible to make them equal depends on the scaling freedom we have of the $E_\alpha$, and the freedom of the choice of basis of $\mf{h}$. We now draw 
\eq{
\ad(E_\alpha) &= \left( \begin{array}{ccc|c|c|ccc}
	0 & \cdots & 0 & 0 & c\pround{\kappa^{-1}_{i1}\alpha^i} & 0 & \cdots & 0 \\ \vdots & \ddots & \vdots & \vdots & \vdots & \vdots & \ddots &\vdots \\ 0 & \cdots & 0 & 0 & c\pround{\kappa^{-1}_{ir}\alpha^i} & 0 & \cdots & 0 \\ \hline -\alpha^1 & \cdots & -\alpha^r & 0 & 0 & 0 & \cdots & 0 \\ \hline 0 & \cdots & 0 & 0 & 0 & 0 & \cdots & 0  \\ \hline 0 & \cdots  & 0 & 0 & 0 & &&\\ \vdots & \ddots & \vdots & \vdots & \vdots & & M_{\beta \gamma} &  \\ 0& \cdots & 0 & 0 & 0 & &&  
\end{array}  \right) \\
&= \left( \begin{array}{c|c|c|c}
 0_{ij} & 0_j & c\pround{\kappa^{-1}_{ij}\alpha^i} & 0_{\beta j} \\ \hline -\alpha^i & 0 & 0 & 0_\beta \\ \hline 0_i & 0 & 0  & 0_\beta \\ \hline 0_{i\gamma} & 0_\gamma & 0_\gamma & M_{\beta\gamma}
\end{array}  \right)
}
where\begin{itemize}
	\item $M$ has only non-zero entries $M_{\beta,\beta+\alpha} = N_{\alpha,\beta}$ when $\alpha+\beta$ is a root
	\item we have ordered our CW basis as $(H_i)_{1 \leq i \leq r}, E_\alpha,E_{-\alpha}, (E_\beta)_{\beta \in \Phi \setminus \pbrace{\pm 1}}$
	\item $c= \kappa\pround{E_\alpha,E_{-\alpha}}$.
\end{itemize}
We therefore see that by choosing a basis of $\mf{h}$ that diagonalises $\kappa$ and then removing any residual scaling through the effect on $c$ of scaling $E_{\pm \alpha}$, we can make the upper left hand s.t. $\ad(E_\alpha)^\dagger = \ad(E_{-\alpha})$. To complete we need to know what happens to $M$. What we need is that $N_{\alpha,\gamma} = N_{-\alpha,\gamma}$. Simple manipulation using the Jacobi rule gives us that 
\eq{
N_{\alpha,\gamma} N_{-\alpha,\alpha+\gamma} + N_{-\alpha,\gamma} N_{\alpha,\gamma-\alpha} = c(\gamma,\alpha)
} 
so we see that if we normalise the basis vectors corresponding to the simple roots first, we may inductively proceed the basis vectors corresponding to higher roots to get the desired result. 
	\begin{comment}
	Recall we define the Hermitian conjugate as a dual wrt the inner product we have, which in this case is the Killing form. Explicitly, we define $\ad(E_\alpha)^\dagger$ by 
	\eq{
\forall X,Y \in \mf{g}, \quad \kappa \pround{\ad(E_\alpha)X,Y} = \kappa\pround{X,\ad(E_\alpha)^\dagger Y}	
}
However the ad-invariance of the Killing form means that 
\eq{
\forall X,Y \in \mf{g}, \quad \kappa \pround{\ad(E_\alpha)X,Y} = - \kappa \pround{X,\ad(E_\alpha)Y}
}
Due to the non-degeneracy of $\kappa$ to complete the proof we need only show that $\ad(E_{-\alpha})$ is a candidate. \hl{solve this problem}
\end{comment}
\end{proof}


\begin{prop}
	$1+1$-dimsensional Toda field theory has a zero curvature representation. 
\end{prop}
\begin{proof}
With $ds^2=dt^2-dx^2=-dx\sp+dx\sp-$, $x\sp\pm=x\pm t$, $\partial_x=\partial_++\partial_-$,
$\partial_t=\partial_+-\partial_-$ the field equations become
$$-\partial_{+-}\phi +\frac{b}4\,[e^{b\phi}E  e^{-b\phi}, E\sp\dagger]=0$$
which are the consistency of
$$0=[\partial_++A_+, \partial_-+A_-],\quad
A_+=\frac{b}2 e^{b\phi/2}\,E \, e^{-b\phi/2}+\frac{b}2 \partial_+\phi,
\quad 
A_-=\frac{b}2 e^{-b\phi/2}\,E\sp\dagger \, e^{b\phi/2}-\frac{b}2 \partial_-\phi
.
$$
To see this recall that $
e^{\phi}E_\alpha  e^{-\phi}=e^{\ad_\phi}E_\alpha =e^{\alpha(\phi)}E_\alpha
$, giving
$$
A_+=\frac{b}2 \sum_{\alpha\in\bar\Delta} \sqrt{n_\alpha}\,e^{b\alpha(\phi)/2}E_\alpha
+\frac{b}2 \partial_+\phi,
\quad 
A_-=
\frac{b}2 \sum_{\alpha\in\bar\Delta} \sqrt{n_\alpha}\,e^{b\alpha(\phi)/2}E_{-\alpha}
-\frac{b}2 \partial_-\phi.
$$
Then
\begin{align*}
A_1&=A_++A_-=
\frac{b}2 \partial_0\phi+{b} \sum_{\alpha\in\bar\Delta} \sqrt{n_\alpha}\,e^{b\alpha(\phi)/2}X\sp+_\alpha\\
A_0&=A_+-A_-=
\frac{b}2 \partial_1\phi+{b} \sum_{\alpha\in\bar\Delta} \sqrt{n_\alpha}\,e^{b\alpha(\phi)/2}X\sp-_\alpha
\end{align*}
where $X\sp\pm_\alpha=(E_\alpha \pm E_{-\alpha})/2$. Thus the zero curvature condition is 
\begin{align*}
0&=[\partial_0+A_0,\partial_1+A_1]=\frac{b}2(\partial_0^2-\partial_1^2)\phi
+\frac{b^2}2  \sum_{\alpha\in\bar\Delta} \sqrt{n_\alpha}\,e^{b\alpha(\phi)/2}
\left[\alpha\left( \partial_0\phi\right)\,X\sp+_\alpha-
\alpha\left( \partial_1\phi\right)\,X\sp-_\alpha\right]
\\
&\qquad+\frac{b^2}2  \sum_{\alpha\in\bar\Delta} \sqrt{n_\alpha}\,e^{b\alpha(\phi)/2}\left(
[ \partial_1\phi,  X\sp+_\alpha]  -[ \partial_0\phi,  X\sp-_\alpha]  \right)
+\frac{b^2}2  \sum_{\alpha\in\bar\Delta}n_\alpha \,e^{b\alpha(\phi)} [E_\alpha,E_{-\alpha}]
\intertext{so giving, using $\comm[\del_\mu \phi]{X_\alpha^\pm}=\alpha\pround{\del_\mu \phi} X_\alpha^\mp$,}\\
0&=\partial_\mu\partial\sp\mu\phi+b\sum_{\alpha\in\bar\Delta}n_\alpha \,e^{b\alpha(\phi)}  [E_\alpha,E_{-\alpha}]=\partial_\mu\partial\sp\mu\phi +b\,[e^{b\phi}E  e^{-b\phi}, E\sp\dagger].
\end{align*}
Finally we note 
\eq{
\comm[\del_0+A_0]{\del_1+A_1} = 2\comm[\del_+ + A_+]{\del_- - A_-}
}
\end{proof}

\begin{remark}
Recall we have the field strength tensor defined as 
\eq{
F_{\mu\nu} = \del_\mu A_\nu - \del_\nu A_\mu + \comm[A_\mu]{A_\nu} = \comm[\del_\mu + A_\mu]{\del_\nu +A_\nu}
}
so the fact that the commutators in light cone coordinates are effectively those in standard Minkowski is not surprising. 
\end{remark}

\begin{remark}
	In the zero curvature equation there so far has been no appearance of a spectral parameter. We see that
	taking
	$$X_\alpha\sp\pm = \frac12\left( \zeta\sp{r_\alpha} E_\alpha \pm \zeta\sp{-r_\alpha} E_{-\alpha}\right)$$
	will result in the same equations of motion. Two common choices in the literature are
	\begin{enumerate}
		\item $r_\alpha=1$ for all $\alpha\in\bar\Delta$,
		\item $r_{-\Theta}=1$ and $r_\alpha=0$ for all $\alpha\in\Delta$.
	\end{enumerate}
\end{remark}

%%%%%%%%%%%%%%%%%%%%%%%%%%%%%%%%%%%%%%%%%%%%%%%%%%%%%%
\subsection{The Spectral Curve}
Having now found the zero curvature representation of affine Toda in $1+1$-dimensions we make the following observation:

\begin{prop}
	Assuming that $\del_0 \phi = 0$, the zero curvature equation becomes a Lax equation with $L=A_0, \, M = A_1$.
\end{prop}
\begin{proof}
Then the independence
from the $0$-coordinate gives
$0=[\partial_1+A_1, \partial_0+A_0]= \partial_1 A_0 +[A_1, A_0]$	
\end{proof}

With this we can start to calculate the spectral curve, which for notational purposes we will fix to be 
\eq{
\mc{C} = \pbrace{\det\pround{L(\zeta) - \eta I}=0}
}
I have coded in Sage a worksheet that takes the fundamental rep corresponding to a Dynkin diagram and finds the corresponding curve as a function of $q_i = \phi^i, \, p_i = -\dot{\phi}^i$.

\begin{example}
	The simplest example we can consider is the $A_1$ Lie algebra, which has roots $\Phi = \pbrace{\pm \alpha}$, highest root $\Theta=\alpha$. A bit of explicit calculation then finds our Lax matrix to be 
	\eq{
\frac{2}{b}L(\zeta) = -pR(H) + \underbrace{\pround{e^{bq}\zeta^{r_1} - e^{-bq}\zeta^{-r_0}}}_{e(\zeta)}R(E) + \underbrace{\pround{e^{-bq}\zeta^{r_0} - e^{bq}\zeta^{-r_1}}}_{f(\zeta)}R(F)	
}
where we have now made the representation $R$ explicit and called $E = E_\alpha, F = E_{-\alpha}$. We also remark the symmetry $e(\zeta) = -f(\zeta^{-1})$. We will drop the factor of $\frac{2}{b}$ from now on as it can be absorbed wlog into $\eta$. We now recall that the reps of $A_1$ are simple when considered as highest weight reps $R=R_\Lambda$, s.t. $L$ is tridiagonal with (using Wikipedia's notation)
\eq{
a_n &= -p\psquare{\Lambda -2(n-1)} \\
b_n &= es_n, \quad (s_n = (\Lambda - n+1)n \text{ a coefficient from rep theory}) \\
c_n &= f
}
with $1 \leq n \leq \Lambda+1$. Hence it is simple to find that the curve can be written as 
\eq{
N \text{ odd } &\Rightarrow \mc{C} = w\prod_{k \text{ even }}^N (w^2 - k^2 y) \\
N \text{ even } &\Rightarrow \mc{C} = \prod_{k \text{ odd }}^N (w^2 - k^2 y)
}
where $y = ef+p^2$ is a function of $z$.
\end{example}

This reducibility into factors is quite general as the following result (see \cite{McDaniel1992}) shows: 
\begin{prop}
	$\forall \lambda$ a dominant weight, $\exists p_\lambda : \mf{g} \times \mbb{C} \to \mbb{C}$ a polynomial s.t. for $\rho:\mf{g} \to \mf{gl}(V)$ a representation with weight multiplicities $m_\lambda$ 
	\eq{
\det \psquare{\rho(x) - z} = \prod_\lambda \psquare{p_\lambda(x,z)}^{m_\lambda} 	
}
where $\lambda$ runs through dominant weights. 
\end{prop}
\begin{proof}
	For $x \in \mf{g}$ write $x=x_s + x_n$, and let $\mf{h}$ be the CSA wrt the weights are defined. $x_s$ must belong to a Cartan subalgebra, and as all CSAs are conjugate $\exists g \in G$ s.t. $\Ad_g x_s \in \mf{h}$. Define 
	\eq{
p_\lambda(x,z) = \prod_{\gamma \in W \cdot \lambda}	\psquare{\gamma\pround{\Ad_g x_s}-z}
}
where $W$ is the Weyl group acting on dominant weights, and then viewing $\gamma \in \mf{h}^\ast$. The choice of $g$ does not affect $p_\lambda$ as the ambiguity is up to conjugation by an element of the Weyl group. As $x$ and $x_s$ have the same eigenvalues with the same multiplicities we have 
	\eq{
	\det \psquare{\rho(x) - z} = \det \psquare{\rho(x_s) - z} =  \prod_\lambda \psquare{p_\lambda(x,z)}^{m_\lambda} 	
}
It remains to show that $p_\lambda$ is a polynomial. This can be done by induction on the height of the weight $\lambda$.  
\end{proof}

We now present some examples of the spectral curve for some small rank Lie algebras. In each case they are given in terms of some generic coefficients, and a parameter $n = \sum_{\alpha \in \bar{\Delta}} n_\alpha r_\alpha$. It can be proven that this is the form of $n$ by considerings rescaling the basis vectors to set $r_\alpha=0$ for $\alpha \in \Delta$, and seeing the effect on $E_{\Theta}$ through the Chevalley relations.  
\eq{
B_2 :& \quad 0 = w\psquare{w^4 + c_1w^2 + c_2 + 16\pround{z^n + z^{-n}} } \\
C_2 :& \quad 0 = w^4 + c_1 w^2 + c_2 + 4 \pround{z^n + z^{-n}} \\
B_3 :& \quad 0 = w\psquare{w^6 + c_1w^4 + c_2w^2 + c_3 + 32\pround{z^n + z^{-n}} } \\
C_3 :& \quad 0 = w^6 + c_1 w^4 + c_2w^2 + c_3 - 16 \pround{z^n + z^{-n}} \\
D_3 :& \quad 0 = w^6 + c_1w^4 + \psquare{c_2 + 4\pround{z^n + z^{-n}}}w^2 + c_3 \\
B_4 :& \quad 0 = w\psquare{w^8 + c_1w^6 + c_2w^4 + c_3w^2 + c_4 - 64\pround{z^n + z^{-n}} } \\
C_4 :& \quad 0 = w^8 + c_1 w^6 + c_2w^4 + c_3w^2 + c_4 + 64 \pround{z^n + z^{-n}} \\
D_4 :& \quad 0 = w^8 + c_1w^6 + c_2 w^4 +\psquare{c_3 - 8\pround{z^n + z^{-n}}}w^2 + c_4 \\
B_5 :& \quad 0 = w\psquare{w^{10} + c_1w^8 + c_2w^6 + c_3w^4 + c_4w^2 + c_5  + 128\pround{z^n + z^{-n}} } \\
C_5 :& \quad 0 = w^{10} + c_1 w^8 + c_2w^6 + c_3w^4 + c_4w^2 + c_5 - 256 \pround{z^n + z^{-n}} \\
D_5 :& \quad 0 = w^{10} + c_1w^8 + c_2 w^6 + c_3 w^4 +\psquare{c_4 + 16\pround{z^n + z^{-n}}}w^2 + c_5
}
Note here I have written the coefficient of the $z^n$ terms, which always turns out to be just an integer, and so we can scale it away in $w$ if we wish. 
\begin{remark}
	When calculating these, I used Sage to do the symbolic calculations. When using the symbolic ring for matrices of size 8 or higher, the calculation was incorrect, and a more specialised ring was required. The reason for this is not yet understood, but lies in how Sage interplays with Maxima. To see this, know that Sage uses Maxima's \texttt{charpoly} to get the determinant, as can be seen from files like \texttt{matrix\_symbolic\_dense.pyx}. To debug further, one would need to look in \texttt{interfaces/maxima.py} to see how the maxima interface handles the matrices. 
\end{remark}
\noindent{\textbf{Questions:}}
\begin{enumerate}
	\item What is the effect on the spectral curve of the different scalings $r_\alpha$? Are the curves birational?
	\item We have an action of the Weyl group on the roots, so does this preserve the spectral curve, or what is its effect on the curve? 
\end{enumerate}

%%%%%%%%%%%%%%%%%%%%%%%%%%%%%%%%%%%%%%%%%%%%%%%%%%%%%%
\subsection{Perturbative Theory}
To make contact with perturbative affine Toda theory we note the expansion
\begin{align*}
\tr e^{b\phi}E  e^{-b\phi} E\sp\dagger&=\tr
(1+b\phi+\frac{b^2}{2}\phi^2+\frac{b^3}{6}\phi^3+\ldots) E
(1-b\phi+\frac{b^2}{2}\phi^2-\frac{b^3}{6}\phi^3+\ldots) 
E\sp\dagger\\
&=\tr \left(E E\sp\dagger+b\phi[E,E\sp\dagger]+\frac{b^2}{2}\phi[ E,[ E\sp\dagger, \phi]]+
\frac{b^3}{6}\phi[ [\phi, E\sp\dagger], [\phi,E]]
+ \ldots
\right)\\
&= \tr E E\sp\dagger+\frac{b^2}{2}\tr \phi[ E,[ E\sp\dagger, \phi]]+ \frac{b^3}{6}\tr \phi[ [\phi, E\sp\dagger], [\phi,E]]+\ldots
\end{align*}
which is further simplified upon specifying the normalisations $ \tr E_\alpha E_{-\alpha}$. This form of the
affine Toda equation has been chosen so that $\phi=0$ is a classical solution. If we work with
$$\tr E_\alpha E_{-\alpha}=\epsilon_\alpha^{-1}
$$
then
$$ \tr E E\sp\dagger=\sum_{\alpha\in\bar\Delta}n_\alpha\sp\vee =g,\qquad  n_\alpha\sp\vee :=
n_\alpha/\epsilon_\alpha,$$
where $g$ is the dual Coxeter number. If we work with the (unshifted) Lagrangian
$$
\mathcal{L}= \frac12 \partial_\mu\psi \partial\sp\mu\psi -
\sum_{\alpha\in\bar\Delta}\epsilon_\alpha e\sp{(\alpha,\psi)}
$$
and expand $\psi=\psi\sp{i}\epsilon_i\lambda_i$ with $(\alpha_i\sp\vee,\lambda_j)=\delta_{ij}$ for the simple roots, then we obtain equations of motion
$$\epsilon_i(\lambda_i,\lambda_j)\epsilon_j \partial_\mu\partial\sp\mu\psi\sp{j}=-
\sum_{\alpha\in\bar\Delta}\epsilon_\alpha (\alpha, \epsilon_i\lambda_i)\, e\sp{(\alpha,\psi)}
= -\epsilon_i\, e\sp{\psi\sp{i}}+n_i \epsilon_{-\Theta }\,e\sp{-(\Theta,\psi)}.
$$
Then with $K_{ij}=(\alpha_i\sp{\vee},\alpha_j)=\epsilon_i (\alpha_i,\alpha_j):=\epsilon_i b_{ij}$ and
$(\lambda_i,\lambda_j)=G_{ij}=\epsilon_i\sp{-1}b_{ij}\sp{-1}\epsilon_j\sp{-1}=\epsilon_i\sp{-1}K_{ij}\sp{-1}$ we obtain
$$-\partial_\mu\partial\sp\mu\psi\sp{j}=b_{ji}\epsilon_i\, e\sp{\psi\sp{i}}-b_{ji}n_i \epsilon_{-\Theta }\,e\sp{-(\Theta,\psi)}
={\bar K}_{ji}\sp{T}\, e\sp{\psi\sp{i}}+{\bar K}_{ji}\sp{T}\, e\sp{-(\Theta,\psi)}
={\bar K}_{ja}\sp{T}\, e\sp{\psi\sp{a}}
$$
and $\psi\sp0:=-(\Theta,\psi)$.


%%%%%%%%%%%%%%%%%%%%%%%%%%%%%%%%%%%%%%%%%%%%%%%%
%%%%%%%%%%%%%%%%%%%%%%%%%%%%%%%%%%%%%%%%%%%%%%%%%


\section{Monopoles and Toda}
From the starting point now of an affine Toda field theory in $1+0$ dimensions (now with time coordinate $s$) as discussed above, real s.t. $\phi=\phi^\dagger$, now define $T_i$ by 
$$
\beta=T_1+i T_2=e^{\phi/2} E e^{-\phi/2},\quad
\beta\sp\dagger=-T_1+i T_2=e^{-\phi/2} E\sp\dagger e^{\phi/2},\quad
\alpha+\alpha\sp\dagger=2i T_3 =\dot \phi.
$$

\begin{prop}
	The $T_i$ satisfy $T_i^\dagger = -T_i$ and Nahm's equations are the Toda equations, i.e.
\begin{align*}
\def\arraystretch{3.2}
\ddot\phi&=[ e^{\phi}E  e^{-\phi} ,E\sp\dagger ] 
\Longleftrightarrow  \begin{cases}
\frac{d}{ds}\beta - \comm[\alpha]{\beta} = 0 \\
\frac{d}{ds}(\alpha + \alpha)^\dagger - \pround{\comm[\alpha]{\alpha^\dagger} + \comm[\beta]{\beta^\dagger}} = 0
\end{cases}
\end{align*}
\end{prop}
\begin{proof}
We first note that if we define $g = e^{-\phi/2}$ then 
\eq{
\alpha = -g^{-1} \dot{g}, \quad \beta = g^{-1} E g
}
and this is the form of the general solution to the complex equation. Next 
	$$[\beta, \beta\sp\dagger]=e^{-\phi/2}[ e^{\phi}E  e^{-\phi} ,E\sp\dagger ]  e^{\phi/2}, \quad \comm[\alpha]{\alpha^\dagger}=0
	$$
so we get the equivalence of the equations. 
\end{proof}

\begin{remark}
	In the above, we are hiding an important point: elements of the algebra are not matrices, and so we need to be actually considering a representation. $e^{\phi/2}Ee^{-\phi/2}$ is actually an algebra element in and of itself recall, as $\Ad_{e^{\phi}} = e^{\ad_\phi}$, but we still need a rep to make it into a matrix. A question I have is what reps $\rho$ ensure that $\rho(E)^\dagger = \rho(E^\dagger)$? Note moreover that we have chosen an algebra $\mf{g}$ s.t. there are $\rank \mf{g}$ fields $\phi$. This ensures that we can have all $\phi$ commuting.  
\end{remark}


\begin{remark}
This coincides with the notation of \emph{Cyclic Monopoles, Affine Toda and Spectral Curves} 
\cite{Braden2011}
\begin{align}
T_1+iT_2&=\begin{pmatrix} 0&e\sp{(q_1-q_2)/2}&0&\ldots&0\\
0&0&e\sp{(q_2-q_3)/2}&\ldots&0\\
\vdots&&&\ddots&\vdots\\
0&0&0&\ldots&e\sp{(q_{n-1}-q_n)/2}\\
e\sp{(q_n-q_1)/2}&0&0&\ldots&0
\end{pmatrix}\\
T_1-iT_2&=-\begin{pmatrix}0&0&\ldots&0&e\sp{(q_n-q_1)/2}\\
e\sp{(q_1-q_2)/2}&0&\ldots&0&0\\
0&e\sp{(q_2-q_3)/2}&\ldots&0&0\\
\vdots&&\ddots&&\vdots\\
0&0&\ldots&e\sp{(q_{n-1}-q_n)/2}&0
\end{pmatrix}\\
T_3&=-\frac{i}{2}\begin{pmatrix} p_1&0&\ldots&0\\
0&p_2&\ldots&0\\
\vdots&&\ddots&\vdots\\
0&0&\ldots&p_n
\end{pmatrix}
\end{align}
where $p_i$, $q_i$ are real in the case of the Lie algebra $A_n$. 
\end{remark}
Upon using $0=\tr E^2=\tr \dot \phi(\beta-\beta\sp\dagger)$
$$\frac12\tr L^2=\frac12\tr\left[\beta -(\alpha+\alpha\sp\dagger)\zeta-\beta\sp\dagger \zeta^2\right]^2
=\zeta^2\tr\left(\frac12 {\dot\phi}^2 -e^{\phi}E  e^{-\phi} E\sp\dagger \right):=\zeta^2 H
$$
and this Hamiltonian is not bounded below\footnote{Here the Lagrangian is
	$\mathfrak{L}:=\tr\left(\frac12 {\dot\phi}^2 +e^{\phi}E  e^{-\phi} E\sp\dagger \right)$ corresponding to a potential of the wrong sign (see the expansion below).}. This is necessary as the monopole boundary conditions require $T_a\sim \rho_a/s $ as $s\sim 0$ (and similarly at $s\sim 2$), where $\rho_a$
is an irreducible $n$-dimensional representation of $su(2)$, thus the momenta are unbounded for
$s\sim 0$ and so the potential must also be unbounded below.



Observe that the Lax matrix for the monopoles may be written 
\begin{align*}
L/\zeta&= -\dot\phi +e^{\phi/2}\,Ee^{-\phi/2}/\zeta-e^{-\phi/2}\,E\sp\dagger e^{\phi/2}\zeta
= -\dot\phi +
\sum_{\alpha\in\bar\Delta}\sqrt{n_\alpha}\, e^{b\alpha(\phi)/2} \left(
\zeta\sp{-1} E_\alpha - \zeta E_{-\alpha}
\right)\\
&=-2 A_0\sp\dagger\\
M&=-\frac12 \dot\phi -e^{-\phi/2}\,E\sp\dagger e^{\phi/2}\zeta
=\frac12 \frac{L}{\zeta} -\frac12 e^{\phi/2}\,\frac{E}{\zeta}e^{-\phi/2}-\frac12 e^{-\phi/2}\,E\sp\dagger e^{\phi/2}\zeta\\
&=\frac12 \frac{L}{\zeta} -\frac12 \sum_{\alpha\in\bar\Delta}\sqrt{n_\alpha}\, e^{b\alpha(\phi)/2} \left(
\zeta\sp{-1} E_\alpha + \zeta E_{-\alpha}
\right)\\
&=-A_0\sp\dagger - A_1\sp\dagger
\end{align*}
and where $\partial_0\phi=0$ and $\partial_1\phi=\dot\phi$ in the previous section. Then the independence
from the $0$-coordinate gives
$0=[\partial_1+A_1, \partial_0+A_0]= \partial_1 A_0 +[A_1, A_0]$ and
$0=\partial_1 A_0\sp\dagger -[A_1\sp\dagger, A_0\sp\dagger] = [\partial_1 -A_1\sp\dagger, A_0\sp\dagger] $
and hence the Lax equation $0=[\partial_1 +M,L]$.

\noindent{\textbf{Questions:}}
\begin{enumerate}
	\item What is the effect on the spectral curve of the different scalings $r_\alpha$? Are the curves birational?
	\item What is the analogue of the characteristic polynomial and determinant for the matrices
	$$a\cdot H+\sum_{\alpha\in\bar\Delta} \left(b_\alpha E_\alpha+c_{\alpha}E_{-\alpha}\right)?$$
	(We may view these as generalizations of tridiagonal matrices.)
\end{enumerate}

%%%%%%%%%%%%%%%%%%%%%%%%%%%%%%%%%%%%%%%%%%%%%%%%%%%%%%%%
%%%%%%%%%%%%%%%%%%%%%%%%%%%%%%%%%%%%%%%%%%%%%%%%%%%%%%%%
\bibliographystyle{../../bib/custom-bib-style}
%\bibliography{../../bib/library,../../bib/manual}
\bibliography{../../bib/jabref_library.bib}
\end{document}
