\documentclass{article}
\usepackage{../../header}

\title{Jordan Normal Form}
\author{Linden Disney-Hogg}
\date{June 2020}

\begin{document}

\maketitle

%%%%%%%%%%%%%%%%%%%%%%%%%%%%%%%%%%%%%%%%%%%%%%%%%%
%%%%%%%%%%%%%%%%%%%%%%%%%%%%%%%%%%%%%%%%%%%%%%%%%%
\section{Introduction}
The purpose of this document is to ask about the density of the Jordan normal form of matrices in the space of matrices. This is motivated by the following result:
\begin{theorem}
	The subset of diagonalisable matrices is dense in $M_n(\mbb{C})$. 
\end{theorem}
Before proving this, we note that I have not told you what topology to take on $M_n(\mbb{C})$. We take the metric topology induced by having a norm $\norm{\cdot}$. That this is well defined requires a bit of theory\footnote{I will assume that you know what a norm is, what the $l_p$-norm is, and what the triangle inequality is. }

\begin{definition}
	Let $V$ be a real or complex vector space. Two norms $\norm{\cdot}_a, \norm{\cdot}_b$ are \bam{equivalent} if $\exists C_1, C_2 \in \mbb{R}_{>0}$ s.t. 
	\eq{
\forall v \in V, \, \left\lbrace \begin{array}{c} \norm{v}_a \leq C_1 \norm{v}_b \\ \norm{v}_b \leq C_2 \norm{v}_a \end{array} \right.	
} 
\end{definition}

\begin{lemma}
	Equivalence of norms is an equivalence relation. 
\end{lemma}

\begin{lemma}
	Let $V$ be a f.d. vector space. Any norm is continuous with respect to the $l_1$-norm
\end{lemma}
\begin{proof}
Taking a basis $\pbrace{e_i}_{i=1}^n$. We can calculate
	\eq{
\forall v \in V, \, \norm{v} &= \norm{v_1 e_1 + \dots +v_n e_n} \\
&\leq \norm{e_1}\abs{v_1} +\dots +\norm{e_n}\abs{v_n} \\
& \leq C \norm{v}_1  \text{ where } C= \max \norm{e_i} \\
\Rightarrow \abs{\norm{v} - \norm{w}} &\leq \norm{v-w} \leq C \norm{v-w}_1
}
\end{proof}

This has the corollary

\begin{theorem}
	If $V$ is a real or complex finite-dimensional vector space, then any two norms on $V$ are equivalent 
\end{theorem}
\begin{proof}
It is sufficient to show that any norm is equivalent to $l_1$. Consider $S_1 = \pbrace{v \in V \, | \, \norm{v}_1=1}\subset V$ and the image of the norm $\norm{\cdot}$ acting on $S_1$. As $S_1$ is compact and $\norm{\cdot}$ is continuous we must have that the image is comapct in $\mbb{R}$, so has a maximmum and minimun obtained $C_1, C_2$. Hence 
\eq{
\forall v \in V, \, C_2 \leq \norm{\frac{v}{\norm{v}_1}} &\leq C_1 \\
\Rightarrow C_2 \norm{v}_1 \leq \norm{v} &\leq C_1 \norm{v}_1
} 
and so we are done. 
\end{proof}
Now we can get to the proof of the theorem we care about
\begin{proof}[Proof of Theorem 1.1]
	The theorem is trivial is $n=1$ so we assume $n\geq 2$. \\
	Let us call $X$ the subset of diagonalisable matrices. Note we have $Y\subset X$ where $Y$ is the set of matrices with distinct eigenvalues. It will suffice to show $Y \subset M_n(\mbb{C})$ is dense. \\
	We now construct the map 
	\eq{
		p:M_n(\mbb{C}) &\to P_n, \\ A &\mapsto p_A
	} from matrices into polynomials of degree $n$ where $p_A$ is the characteristic polynomial defined by 
	\eq{
p_A(t) = \det (tI-A)	
}
This map is continuous (viewing $P_n$ inside the vector space of polynomials of degree $\leq  n$, and inheriting the metric). We also have the continuous map $\Delta : P_n \to \mbb{R}$ given by the discriminant. Moreover, the composed map $\Delta \circ p : M_n(\mbb{C})\to \mbb{R}$ is a non-constant polynomial in the components, and so cannot be locally constant. Recall that the discriminant of a polynomial is 0 iff it has a repeated root, and a repeated root in a characteristic polynomials corresponds to a an eigenvalue with algebraic multiplicity $>1$ of the original matrix. Hence we have 
\eq{
\forall A \in M_n(\mbb{C}) \setminus Y, \,  (\Delta \circ p)(A) = 0 
}
and if $U\ni A$ is an open neighbourhood 
\eq{
	\exists d \in (\Delta \circ p)(U), \, d \neq 0
}
We thus know any neighbourhood of $A$ intersects $Y$ non-trivially, and so $Y$ is dense. 
\end{proof}

%%%%%%%%%%%%%%%%%%%%%%%%%%%%%%%%%%%%%%%%%%%%%%%%%%%%%%%%
%%%%%%%%%%%%%%%%%%%%%%%%%%%%%%%%%%%%%%%%%%%%%%%%%%%%%%%%
\section{Jordan normal form}
We now want to see if we can extend this idea to ask about when we know an eigenvalue has algebraic multiplicity $\geq 1$, whether there is a subset of these matrices that is dense in the matrices with a given multiplicity. 

\begin{theorem}
 The Jordan normal form of a matrix exists and is unique up to reordering of jordan blocks. 
\end{theorem}
We can define an equivalence relation on Jordan form matrices if they are the same up to a reordering of block.
We can then fix notation to let $J:M_n(\mbb{C}) \to M_n(\mbb{C})$ be the map that sends a matrix to it's Jordan normal form equivalence class. That such a map exists and is well defined is a result of the above theorem. We can then biject between the image of $J$ and the cosets of $M_n(\mbb{C})$ quotiented by $GL_n(\mbb{C})$ acting by conjugation. \\
We will also fix notation as such: 
\begin{definition}
The \bam{Jordan block of size m} for $m \geq 2$ is the $m \times m$ matrix given by  
\eq{
J_m(\lambda) = \lambda I_{m \times m} + E_{12} + \dots + E_{m-1,m} 
}
where $E_{ij}$ is the standard basis of $M_m(\mbb{R})$
\end{definition}
A definition we will also want that will be useful is 
\begin{definition}
	The \bam{centraliser} of a matrix $A \in M_n(\mbb{C})$ is 
	\eq{
C(A) = \pbrace{B \in M_n(\mbb{C}) \, | \, \comm[A]{B}=0}	
}
\end{definition}

\begin{lemma}
	$B \in C(A)\cap GL_n(\mbb{C}) \Leftrightarrow B^{-1}AB = A$
\end{lemma}

\begin{lemma}
	$C(A) \cap GL_n(\mbb{C}) \leq GL_n(\mbb{C})$
\end{lemma}

I will often consider the restriction of the centraliser to the invertible elements, as this will be important from here on in with regards to Jordan normal form.

%%%%%%%%%%%%%%%%%%%%%%%%%%%%%%%%%%%%%%%%%%%%%%%%%%%%%%%%
\subsection{\secmath{n=2}}
Let us start with an easy question we can make traction on, the case of $2 \times 2$ matrices. We can split the Jordan normal form of such matrices into three groups 
\begin{center}
	$
	\begin{array}{ccc}
		A & p_A & m_A \\
		\hline \hline 
	 	\begin{psmallmatrix} \lambda & 0 \\ 0 & \lambda \end{psmallmatrix} & (t-\lambda)^2 & (t-\lambda) \\
	 			\begin{psmallmatrix} \lambda & 1 \\ 0 & \lambda \end{psmallmatrix} & (t-\lambda)^2 & (t-\lambda)^2 \\
	 					\begin{psmallmatrix} \lambda & 0 \\ 0 & \mu \end{psmallmatrix} & (t-\lambda)(t-\mu) & (t-\lambda)(t-\mu)
	\end{array}
	$
\end{center}
Here we have also shown the characteristic polynomial $p_A$ and the minimal polynomial $m_A$. We see what we know to be true, that the characteristic polynomial is not sufficient to distinguish between all cases of Jordan normal form

\begin{remark}
In the $2 \times 2$ case the minimal polynomial along is sufficient to characterise, that is there is a bijection between $m_A \leftrightarrow J(A)$. In the $3 \times 3$ case there is a bijection $(m_A,p_A) \leftrightarrow J(A)$. In higher cases you need more information that the minimal and characteristic polynomial. 
\end{remark}

In this case we see that, if we fix the diagonal to be $\lambda, \lambda$, then there is only one matrix with the Jordan normal form $\begin{psmallmatrix} \lambda & 0 \\ 0 & \lambda \end{psmallmatrix}$ (namely $\begin{psmallmatrix} \lambda & 0 \\ 0 & \lambda \end{psmallmatrix}$ itself, as it is invariant under conjugation). Contrastingly, the only matrices which preserve $\begin{psmallmatrix} \lambda & 1 \\ 0 & \lambda \end{psmallmatrix}$ under conjugation are those that commute with it. Let's calculate these conditions:
\eq{
\begin{pmatrix} a & b \\ c & d \end{pmatrix} \begin{pmatrix} \lambda & 1 \\ 0 & \lambda \end{pmatrix}&= \begin{pmatrix} \lambda & 1 \\ 0 & \lambda \end{pmatrix}\begin{pmatrix} a & b \\ c & d \end{pmatrix} \\
\Rightarrow \begin{pmatrix} \lambda a & a+\lambda b \\ \lambda c  & c+\lambda d \end{pmatrix}&= \begin{pmatrix} c+ \lambda a & d + \lambda b \\ \lambda c & \lambda d \end{pmatrix} \\
\Rightarrow c &= 0, \, d = a
}
The condition that the matrix $\begin{psmallmatrix} a & b \\ c & d \end{psmallmatrix}$ is invertible then gives 
\eq{
C\pround{\begin{psmallmatrix} \lambda & 1 \\ 0 & \lambda \end{psmallmatrix}} = \pbrace{\begin{psmallmatrix} a & b \\ 0 & a \end{psmallmatrix} \, | \, a,b \in \mbb{C}, \, a \neq 0}
}
Note here that the centraliser of the Jordan block is independent of the value on the diagonal. This is in fact general

\begin{prop}
	$C(J_m(\lambda)) = C(J_m(0))$
\end{prop}
In order to think about matrices with a fixed set of eigenvalues, we need to understand isospectral deformations. 








\section{What Harry Says}



\subsubsection{Some Linear Algebra}

Its useful to gather some properties of commutators of matrices in Jordan form. Let
$\tilde L=\diag(J_{\mu_1}(\lambda_1),\ldots, J_{\mu_t}(\lambda_t) )$ be written in Jordan form. Here
$J_{\mu_i}(\lambda_1)$ is a $\mu_i\times\mu_i$ Jordan block with diagonal $\lambda_i$ and $1$'s just 
above the diagonal and we are not assuming the $\lambda_i$ distinct. In considering the matrix
$[{\tilde L},{\tilde M}]$ we will write $\tilde M$ as $\mu_i\times \mu_j$ blocks and denote by $M'$ the $i$-th diagonal block of $\tilde M$. We wish to solve
\begin{align}
[{\tilde L},{\tilde M}]&=0.
\end{align}
First, we
see for the $i$-th diagonal block $[J_{\mu_1}(\lambda_1),M']=0$. Solving this we see $M'$ is upper triangular with constant value along the $M'_{i,i+k}$ diagonals ($k=0,\dots,n-1$); for example with $\mu_i=3$,
$$M'=\begin{pmatrix}x&y&z\\ 0&x&y\\ 0&0&x \end{pmatrix}.$$
For the off-diagonal blocks of $\tilde M$ consider $J_{\mu_i}(\lambda_i)B-
B J_{\mu_j}(\lambda_j)=0$, where $B$ is a $\mu_i\times \mu_j$ matrix. There are two cases.
If $\lambda_i\ne \lambda_j$ then starting with the $\mu_i$-th row and $1$-st column we find
$B_{\mu_i1}(\lambda_i- \lambda_j)=0$ and so $B_{\mu_i1}=0$; recursively substituting we find
$B=0$. If $\lambda_i= \lambda_j$ we get a block of forms
$$\begin{pmatrix}0&0&x&y&z\\ 0&0&0&x&y\\ 0&0&0&0&x \end{pmatrix},\quad
\begin{pmatrix}x&y&z\\ 0&x&y\\ 0&0&x\\ 0&0&0\\  0&0&0\end{pmatrix},
$$
depending on whether $\mu_i<\mu_j$ or $\mu_j<\mu_i$ respectively. 
In either case we find this space has dimension $\min(\mu_i,\mu_j)$.
Note that when we have the
$m$ Jordan blocks of size $1$ associated with the same eigenvalue these combine to give a $m\times m$ matrix.

\noindent{\textbf {Example}:}


$$
\begin{pmatrix} \lambda&0&0&0&0&0&0&0&0
\\ \noalign{\medskip}0&\lambda&0&0&0&0&0&0&0\\ \noalign{\medskip}0&0&\lambda&1&0&0
&0&0&0\\ \noalign{\medskip}0&0&0&\lambda&1&0&0&0&0\\ \noalign{\medskip}0&0&0
&0&\lambda&0&0&0&0\\ \noalign{\medskip}0&0&0&0&0&\lambda&1&0&0
\\ \noalign{\medskip}0&0&0&0&0&0&\lambda&1&0\\ \noalign{\medskip}0&0&0&0&0&0
&0&\lambda&1\\ \noalign{\medskip}0&0&0&0&0&0&0&0&\lambda
\end{pmatrix},
\qquad
\begin{pmatrix}m_{{1,1}}&m_{{1,2}}&0&0&a&0&0&0&e
\\ \noalign{\medskip}m_{{2,1}}&m_{{2,2}}&0&0&b&0&0&0&f
\\ \noalign{\medskip}c&d&x_{{1}}&x_{{2}}&x_{{3}}&0&\alpha&\beta&\gamma
\\ \noalign{\medskip}0&0&0&x_{{1}}&x_{{2}}&0&0&\alpha&\beta
\\ \noalign{\medskip}0&0&0&0&x_{{1}}&0&0&0&\alpha\\ \noalign{\medskip}
g&h&\rho&\sigma&\tau&y_{{1}}&y_{{2}}&y_{{3}}&y_{{4}}
\\ \noalign{\medskip}0&0&0&\rho&\sigma&0&y_{{1}}&y_{{2}}&y_{{3}}
\\ \noalign{\medskip}0&0&0&0&\rho&0&0&y_{{1}}&y_{{2}}
\\ \noalign{\medskip}0&0&0&0&0&0&0&0&y_{{1}}
\end{pmatrix}.
$$

First recall that a matrix $A$ is similar to the matrix $J$ if and only if there exists an invertible
matrix $T$ such that $A=T J T\sp{-1}$. Then if $A=T J T\sp{-1}=T_1 J T_1\sp{-1}$ we have
$S:=T_1\sp{-1} T$ satisfies $S J=JS$, giving
\begin{lemma} If $A=T J T\sp{-1}$ then $A=T_1 J T_1\sp{-1}$ if and only if $S:=T_1\sp{-1} T$ and
	\begin{equation}
	S J=JS.
	\end{equation}
	Further,
	\begin{equation}
	\dim\{A\,|\, A=T J T\sp{-1}\}=\dim\{T\,|\, TT\sp{-1}=1\}-
	\dim\{S\,|\, SS\sp{-1}=1,\ SJ=JS\}.
	\end{equation}
\end{lemma}
This may be used 
to calculate the dimensions of various standard classes of matrices.

Let us work over $\mathbb{C}$ for simplicity. Here the dimension of the space of invertible matrices is
$n^2=\dim\{T\,|\, T\in M_n(\mathbb{C}),\ TT\sp{-1}=1\}$ and the dimension of the set of matrices with distinct eigenvalues of  specified multiplicities  $\mu_i$ ($i=1,\ldots,t$) has dimension $t$.

\begin{lemma} The set of diagonalizable matrices in $M_n(\mathbb{C})$ with (specified) distinct eigenvalues
	$(\lambda_1,\ldots,\lambda_t)$ of multiplicities $\mu_i$ ($i=1,\ldots,t$) has (complex) dimension
	$$n^2 - \sum_{i=1}\sp{t} \mu_i^2.$$
	The dimension of the set of matrices with distinct eigenvalues of  specified multiplicities 
	$\mu_i$ ($i=1,\ldots,t$) has (complex) dimension
	$$n^2 - \sum_{i=1}\sp{t}( \mu_i^2-1).$$
\end{lemma}

We have seen when discussing matrices that commute with a matrices in Jordan form that if
$B$ is a $\mu_i\times \mu_j$ matrix such that $J_{\mu_i}(\lambda_i)B-
B J_{\mu_j}(\lambda_j)=0$, then $B=0$ if $\lambda_i\ne \lambda_j$. Thus when counting
the dimension of matrices commuting with a matrix in  Jordan form  we can restrict attention to those matrices with a common eigenvalue, and then sum over the distinct eigenvalues. Let
$$J=\diag(J_{\mu_1}(\lambda),\ldots, J_{\mu_k}(\lambda) ),\quad \mu_1\ge\mu_2\ge\ldots\ge \mu_k,
$$ 
have $k$ Jordan blocks with common eigenvalue $\lambda$. Then we have
$$\dim\{S\,|\, SS\sp{-1}=1,\ SJ=JS\}= \sum_{i,j=1}\sp{k} \min(\mu_i,\mu_j)
= \sum_{j=1}\sp{k} (2j-1)\mu_j.
$$
Now suppose that a matrix has Jordan form $J$ with $p$ distinct eigenvalues $\lambda_a$ of
multiplicities $n_a$ ($a=1,\ldots,p$) each of which is formed of formed from $N_a$ blocks of size
$\mu_{aj}$,
\begin{equation}\label{jformdim}
J=\diag(J_{\mu_{a1}}(\lambda_a),\ldots, J_{\mu_{aN_a}}(\lambda_a) ),\quad
\mu_{a1}\ge\mu_{a2}\ge\ldots\ge \mu_{aN_a}, \quad
\sum_{i=1}\sp{N_a}  \mu_{ai}=n_a.
\end{equation}
Set $N\sp\ast=\max_a n_a$ and $\mu_{al}=0$ for  $l>N_a$. Then
$$\dim\{S\,|\, SS\sp{-1}=1,\ SJ=JS\} 
=\sum_{a=1}\sp{p} \sum_{j=1}\sp{N_a} (2j-1)\mu_{aj}
= \sum_{j=1}\sp{N\sp\ast} (2j-1)\sum_{a=1}\sp{p}\mu_{aj}
$$
\begin{lemma} The set of complex $n\times n$ matrices $A$ with Jordan form (\ref{jformdim}) and prescribed (distinct) eigenvalues $\lambda_a$ is
	\begin{align}
	\dim \{A\,|\, p, \lambda_a N_a,\mu_{aj}\}= n^2- \sum_{a=1}\sp{p} \sum_{j=1}\sp{N_a} (2j-1)\mu_{aj}
	= n^2- \sum_{j=1}\sp{N\sp\ast} (2j-1)\sum_{a=1}\sp{p}\mu_{aj}.
	\intertext{The complex dimension of matrices $A$ with these properties and any 
		(distinct) eigenvalues $\lambda_a$ is}
	\dim \{A\,|\, p,N_a,\mu_{aj} \}= n^2- \sum_{a=1}\sp{p} \sum_{j=1}\sp{N_a} (2j-1)\mu_{aj}+p
	= n^2- \sum_{j=1}\sp{N\sp\ast} (2j-1)\sum_{a=1}\sp{p}\mu_{aj}+p.
	\end{align}
\end{lemma}

\begin{enumerate}[(a)]
	\item When all the eigenvalues of $A$ are simple and distinct then $p=n$, $n_a=N_a=\mu_{1j}=1$,
	$$\dim \{A\,|\, p=n,\   n_1=N_1=1,\  \mu_{11}=1   \}= n^2,$$
	
	\item When  $A$  has one Jordan block with $n$-fold eigenvalue
	then $p=1$, $n_1=n$, $N_1=1$, $\mu_{11}=n$,
	$$\dim \{A\,|\, p=1,\   n_1=n,\ N_1=1,\  \mu_{11}=n    \}= n^2-n+1,$$
	
	\item When  $A$  has one Jordan block with $n$-fold eigenvalue and $n$ blocks
	then $p=1$, $n_1=n$, $N_1=n$, $\mu_{11}=1$,
	$$\dim \{A\,|\, p=1,\   n_1=n,\ N_1=1,\  \mu_{11}=1    \}= 1.$$
	
	\item When  $A$  has $n-1$ distinct eigenvalues and one nontrivial Jordan block with $2$-fold eigenvalue
	then $p=n-1$, $n_1=2$, $N_1=1$, $\mu_{11}=2$,  $1=n_j=N_j=\mu_{j1}$ ($j>1$),
	$$\dim \{A\,|\, p=n-1,\ n_1=2,\ N_1=1,\ \mu_{11}=2,\ 1=n_j=N_j=\mu_{j1}\ (j>1)   \}= n^2-1,$$
	
	\item When  $A$  is diagonalizable with $n-1$ distinct eigenvalues $p=n-1$, $n_1=2$, $N_1=2$, $\mu_{11}=\mu_{12}=1$,  $1=n_j=N_j=\mu_{j1}$ ($j>1$),
	$$\dim \{A\,|\, p=n-1,\ n_1=2,\ N_1=2,\ \mu_{11}=\mu_{12}=1,\ 1=n_j=N_j =\mu_{j1}\ (j>1)   \}= n^2-3.$$
	
\end{enumerate}

We see that when two eigenvalues coincide that the generic situation is for a Jordan block of codimension $1$; the diagonalizable case is of codimension $3$.

\noindent{\textbf {Example}:}
Consider the matrix 
$$\begin{pmatrix}0&1-\epsilon^2\\ -b^2 &2b\end{pmatrix}=\Lambda.
\begin{pmatrix}b+\epsilon b&0\\ 0&b-\epsilon b \end{pmatrix}\Lambda\sp{-1},\quad
\Lambda=\begin{pmatrix}1-\epsilon&1+\epsilon\\  b&b\end{pmatrix}.
$$
As $\epsilon\rightarrow0$ this has Jordan form
$$\begin{pmatrix}b&1\\0&b\end{pmatrix}.$$

\noindent{\textbf {Example}:}
For $3\times3$ matrices we have for distinct $\lambda_i$ that 
\begin{enumerate}[(i)]
	\item $\dim\diag(\lambda_1,\lambda_2,\lambda_3)=9$,
	\item $\dim\diag( J_2(\lambda_1), \lambda_2)=8$, 
	\item $\dim\diag( J_3(\lambda_1))=7$,
	\item $\dim\diag(\lambda_1,\lambda_1,\lambda_2)=6$,
	\item $\dim\diag(\lambda_1,\lambda_1,\lambda_1)=1$.
\end{enumerate}


%%%%%%%%%%%%%%%%%%%%%%%%%%%%%%%%%%%%%%%%%%%%%%%%%%%%%%%%
%%%%%%%%%%%%%%%%%%%%%%%%%%%%%%%%%%%%%%%%%%%%%%%%%%%%%%%%
\bibliographystyle{../../bib/custom-bib-style}
\bibliography{../../bib/library,../../bib/manual}

\end{document}
