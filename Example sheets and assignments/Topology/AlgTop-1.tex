\documentclass{article}

\usepackage{header-colourful}
%%%%%%%%%%%%%%%%%%%%%%%%%%%%%%%%%%%%%%%%%%%%%%%%%%%%%%%%
%Preamble

\title{Algebraic Topology Example Sheet 1}
\author{Linden Disney-Hogg}
\date{November 2019}

%%%%%%%%%%%%%%%%%%%%%%%%%%%%%%%%%%%%%%%%%%%%%%%%%%%%%%%%
%%%%%%%%%%%%%%%%%%%%%%%%%%%%%%%%%%%%%%%%%%%%%%%%%%%%%%%%
\begin{document}

\maketitle
\tableofcontents

%%%%%%%%%%%%%%%%%%%%%%%%%%%%%%%%%%%%%%%%%%%%%%%%%%%%%%%%
%%%%%%%%%%%%%%%%%%%%%%%%%%%%%%%%%%%%%%%%%%%%%%%%%%%%%%%%
\section{Question 1}
Let $Y$ be a space and $\gamma$ a path in $Y$
%%%%%%%%%%%%%%%%%%%%%%%%%%%%%%%%%%%%%%%%%%%%%%%%%%%%%%%%
\subsection{Part i)}
We start by recalling the following definition:
\begin{definition}[Basepoint homomorphism]
Given $\gamma$ there is the associated map 
\eq{
\Phi_{[\gamma]} : \pi_1 \pround{Y,\gamma(1)} \to \pi_1 \pround{Y,\gamma(0)}
}
given by 
\eq{
\Phi_{[\gamma]}\pround{[p]} = [\gamma^{-1}\cdot p \cdot \gamma]
}
where $\cdot$ is the well understood operation of path composition, and $\gamma^{-1}$ is "travelling backwards" along $\gamma$. 
\end{definition}
Note that this is indeed a good definition, as if $p$ is a loop based at $\gamma(1)$, then $\gamma^{-1} \cdot p \cdot \gamma$ travels from $\gamma(0)$ along $\gamma$ to $\gamma(1)$, back to $\gamma(1)$ after travelling along $p$, and then back to $\gamma(0)$ on $\gamma^{-1}$. It is also clearly a homomorphism as
\eq{
(\gamma^{-1} \cdot p \cdot \gamma) \cdot (\gamma^{-1} \cdot q \cdot \gamma) = \gamma^{-1} \cdot (p \cdot q) \cdot \gamma
}
Moreover it has the well defined inverse given by $\Phi_{[\gamma^{-1}]}$, and so it is an iso. 

%%%%%%%%%%%%%%%%%%%%%%%%%%%%%%%%%%%%%%%%%
\subsection{Part ii)}
Now set $f,g : X \to Y$ with a homotopy between them $H$, and define $\gamma = H \rvert_{\pbrace{x_0} \times I}$ for basepoint $x_0 \in X$. Recall we can define 
\eq{
f_\ast : \pi_i(X,x_0) &\to \pi_i(Y,f(x_0)) \\
 [p] &\mapsto [f \circ p]
}
\begin{lemma}
for $i=1$, $f_\ast = \Phi_{[\gamma]} \circ g_\ast$
\end{lemma}
\begin{proof}
Certainly we can see 
\eq{
\pi_1(X,x_0) \overset{g_\ast}{\to} \pi_1(Y,g(x_0)) \overset{\Phi_{[\gamma]}
}{\to} \pi_1(Y,f(x_0))
}
by the previous part. Moreover, for $p:I \to X$ with endpoints at $x_0$, see $f\circ p, g \circ p : I \to Y$ are homotopic by the map 
\eq{
I \times I &\to Y \\
(s,t) &\mapsto H((p(s),t))
}
so it must be the case that 
\eq{
[f \circ p] = [\gamma^{-1} \cdot (g \circ p) \cdot \gamma]
}
as based loops.  
\end{proof}

%%%%%%%%%%%%%%%%%%%%%%%%%%%%%%%%%%%%%%%%%%%
\subsection{Part iii)}
Now let $H$ be a homotopy between $id_X$ and itself, and define $\gamma$ as before. 
\begin{lemma}
$[\gamma] \in Z\pround{\pi_1(X,x_0)}$
\end{lemma}
\begin{proof}
By the above, we must have that $\Phi_{[\gamma]}$ acts as the identity on $\pi_1(X,x_0)$. As such we know 
\eq{
\forall p, \, [p] = [\gamma^{-1} \cdot p \cdot \gamma] = [\gamma]^{-1} [p] [\gamma]
}
i.e. $[\gamma]$ is in the centre. 
\end{proof}

%%%%%%%%%%%%%%%%%%%%%%%%%%%%%%%%%%%%%%%%%%%
%%%%%%%%%%%%%%%%%%%%%%%%%%%%%%%%%%%%%%%%%%%
\section{Question 2}
%%%%%%%%%%%%%%%%%%%%%%%%%%%%%%%%%%%%%%%%%%%
\subsection{Part i)}
We will start with a statement of the CW approximation theorem. For completeness we will recall a few necessary definitions.

\begin{definition}[Weak homotopy equivalence]
A map $f : X \to Y$ is a \bam{weak homotopy equivalence} if $\forall n \geq 0, x_0 \in X$, the map $f_\ast : \pi_n(X,x_0) \to \pi_n(Y,f(x_0))$ is an isomorphism.
\end{definition}

\begin{theorem}[CW approximation]
$\forall X$ a space, $\exists Y$ a CW complex with $f:Y\to X$ a weak homotopy equivalence. 
\end{theorem}
\begin{proof}
We will sketch the proof presented in Hatcher here for completeness. \\
The first step is to note that, given a map from a CW complex $f:A\to X$ which a basepoint $a_i$ for each connected component of $A$ we can construct a new CW complex $B$ by attaching k-cells to $A$ such that $f_\ast$ is injective for $n=k-1$ and surjective for $n=k$, when $f$ is extended to $B$. This is construction is done by 
\begin{itemize}
    \item For $n=k-1$ s.t $f_\ast$ is not injective, take a set of generators for the kernel, indexed by $\alpha$. We may "fill in" the gaps that are causing non-trivial homotopies at each of these generators by choosing an attaching map $\phi_\alpha :(S^{k-1},s_0) \to (A,a_i)$, which intuitively will cover this cycle with a $D^k$, which will then be k-connected. 
    \item For $n = k$ s.t $f_\ast$ is not surjective, choose generators of these cycles, indexed by $\beta$ which will be of the form $f_\beta : S^k \to X$ s.t. $f_\beta$ is not null homotopic. We can then attach k cells to $A$ indexed by $\beta$ and extend $f$ along them by the $f_\beta$. 
\end{itemize}
We now make the observation that if $f: A \to X$ being injective for $n < k-1$ or surjective for $n < k$ will not have changed by making the attachments, as the $\pi_n$ are trivial of $S^k$ for $k > n+1$, and adding cells cannot decrease surjectivity. As such starting with a CW complex with just 0-cells corresponding to the connected components of $X$, a CW complex on which all the $f_\ast$ are bijective can be built inductively. 
\end{proof}

%%%%%%%%%%%%%%%%%%%%%%%%%%%%%%%%%%%%%%%5
\subsection{Part ii)}
We can now state the definition of an Eilenberg Maclane space sensibly:

\begin{definition}[Eilenberg-Maclane]
Given a group $G$, and $n\in\mbb{N}$, an \bam{Eilenberg Maclane space} of type $K(G,n)$ is a topological space $X$ with $\pi_n(X) \cong G$ and all other homotopy groups trivial. 
\end{definition}

%%%%%%%%%%%%%%%%%%%%%%%%%%%%%%%%%%%%%%%5
\subsection{Part iii)}
Now let $X$ be a CW complex with 1 0-cell, $x_0$, and no $k$ cells for $0<k<n$. 

\begin{prop}
X can be embedded into $K(\pi_n(X,x_0),n)$ space.
\end{prop}
\begin{proof}
Note first that we know $\pi_k(X,x_0)$ is trivial for $0 \leq k < n$ by the fact that $\pi_k(S^n)$ is. Then by the CW approximation theorem we can extend $X$ to $Z$ an approximation to the Eilenberg maclane space, adding $k\geq n$ cells to make higher homotopy groups trivial, and we get an immediate embedding by inclusion.
\end{proof}

%%%%%%%%%%%%%%%%%%%%%%%%%%%%%%%%%%%%%%%5
\subsection{Part iv)}
We now wish to construct $\pi_k(\mbb{RP}^n)$ for $n \geq 2, 1 \leq i \leq n$. It will be necessary to use the fact 
\eq{
\pi_k(S^n) = \left\lbrace \begin{array}{cc} 0 & k < n \\ \mbb{Z} & k = n \end{array}  \right.
}
We then know we can construct the fibre bundle $\mbb{Z}_2 \to S^n \to \mbb{RP}^n$, which is in fact a fibration as we have CW complexes, and so we have a long exact sequence 
\eq{
\cdots \to \pi_k(\mbb{Z}_2) \to \pi_k(S^n) \to \pi_k(\mbb{RP}^n) \to \pi_{k-1}(\mbb{Z}_2) \to \cdots \to \pi_0(\mbb{Z}_2) \to \pi_0(S^n)
}
Now identifying 
\eq{
\pi_k(\mbb{Z}_2) = \left\lbrace \begin{array}{cc} 0 & k \geq 1 \\ \mbb{Z}_2 & k = 0 \end{array}  \right.
}
we know that
\eq{
  \pi_1(S^n) \to \pi_1(\mbb{RP}^n) &\to \pi_0(\mbb{Z}_2) \to \pi_0(S^n) \\ 
 0 \to  \pi_1(\mbb{RP}^n) &\to \mbb{Z}_2 \to 0 \\ 
 \pi_1(\mbb{RP}^n) &= \mbb{Z}_2 
}
and 
\eq{
  \pi_k(\mbb{Z}_2) \to \pi_k(S^n) \to \pi_k(\mbb{RP}^n) &\to \pi_{k-1}(\mbb{Z}_2)  \\ 
 0 \to \pi_k(S^n) \to \pi_k(\mbb{RP}^n) & \to 0 \\ 
  \pi_k(S^n) = \pi_k(\mbb{RP}^n) &
}
for $k > 1$ giving 
\eq{
\pi_k(\mbb{RP}^n) = \left\lbrace \begin{array}{cc} \mbb{Z}_2 & k =1 \\ 0 & 1<k<n \\ \mbb{Z} & k=n \end{array}  \right.
}
Observing the above, we note that as per iii) an Eilenberg Maclane space $K(\mbb{Z}_2,1)$ into which $\mbb{RP}^n$ would embed and that would have the correct homotopy groups would be 
\eq{
\mbb{RP}^\infty \equiv \bigcup_{n=1}^\infty \mbb{RP}^n
}
We can recognise that this would not have nontrivial higher homotopy, as any such image of $S^k$ can be contracted to a point in the copy of $\mbb{RP}^{k+1}$ in the union. 
%%%%%%%%%%%%%%%%%%%%%%%%%%%%%%%%%%%%%%%%%%%
%%%%%%%%%%%%%%%%%%%%%%%%%%%%%%%%%%%%%%%%%%%
\section{Question 3}

For completeness here we will state a definitions:

\begin{definition}[Compact Open Topology]
Denote $X^Y$ to be the set of maps $f:Y \to X$. The \bam{compact open topology} on this set is generated by the sets 
\eq{
M_{K,U} = \pbrace{f \rvert f(K) \subset U}
}
for $K \subset Y$ compact and $U \subset X$ open. $X^Y$ with the compact open topology is called the \bam{mapping space}. 
\end{definition}

The Mapping space has the following key property: given a fibration $i : B \to Y$, the induced map $i^\ast : X^Y \to X^B$ is a fibration (under suitable conditions on $B$ and $Y$ which for our purposes will be satisfied). 


%%%%%%%%%%%%%%%%%%%%%%%%%%%%%%%%%%%%%%%%%%%
\subsection{Part i)}
\begin{lemma}
$X^{\{0,1\}} \cong X \times X$
\end{lemma}
\begin{proof}
We can clearly create the map
\eq{
X^{\{0,1\}} &\overset{\phi}{\to} X \times X \\
f &\mapsto (f(0),f(1))
}
which is a bijection with inverse 
\eq{
X \times X &\to X^{\{0,1\}} \\
(x,y) &\mapsto \left\lbrace \begin{array}{cc} 0 \mapsto x \\ 1 \mapsto y \end{array}\right\rbrace
}
As $\pbrace{0,1}$ was given the discrete topology, all subsets are compact and open. So given $U,V \subset X$ open we have 
\eq{
\phi^{-1}(U\times V) = M_{\pbrace{0},U} \cap M_{\pbrace{1},V}
}
hence open, giving $\phi$ continuous. Conversely
\eq{
(\phi^{-1})^{-1}(M_{K,U}) = \phi(M_{K,U}) = U \times U
}
again open, so we have that $\phi$ is a homeomorphism. 
\end{proof}

%%%%%%%%%%%%%%%%%%%%%%%%%%%%%%%%%%%%%%%%%%%
\subsection{Part ii)}
With the above I can make the following proposition
\begin{prop}
The map $\pi_X : X^I \to X \times X$, $\pi_X(\gamma) = (\gamma(0),\gamma(1))$ is a fibration.  
\end{prop}
\begin{proof}
Note that by above the is equivalent to saying that $X^I \to X^{\pbrace{0,1}}, \, \gamma \mapsto \gamma\rvert_{\pbrace{0,1}}$ is a fibration, and by our key property of the mapping space it is sufficient to ask whether $i:\pbrace{0,1} \to I$ is a cofibration. We now use general theory to recall that given a map $f:X \to Y$, the inclusion map $i:X \to M_f$, the mapping cylinder, is a cofibration.  Taking $f = c_0 :\pbrace{0,1} \to \pbrace{0}$, we see $M_f$ is homeomorphic to $I$, and so we are done. 
\end{proof}

%%%%%%%%%%%%%%%%%%%%%%%%%%%%%%%%%%%%%%%%%%%
\subsection{Part iii)}
Let us now define the subspace of $X^I$ 
\eq{
PX = \pbrace{\gamma \rvert \gamma(0) = x_0}
}
for some $x_0 \in X$. 
\begin{prop}
The map 
\eq{
p_X : PX &\to X \\
\gamma &\mapsto \gamma(1)
}
is a fibration. 
\end{prop}
\begin{proof}
We will want to use the fact from exercises that given the commuting diagram \\
\begin{center}
\begin{tikzcd}
A \times_B E \arrow[r, "\bar{f}"] \arrow[d, "q"]
& E \arrow[d, "p"] \\
 A \arrow[r, "f"]
& B
\end{tikzcd} 
\end{center}
for $f:A\to B$, $E\overset{p}{\to}B$ a fibration, $\bar{f},q$ the corresponding projections from 
\eq{
A\times_B E = \pbrace{(a,e) \, \rvert \, f(a) = p(e)}
}
then $q$ is also a fibration. Now note that this can be applied to the above forming the diagram \\
\begin{center}
    \begin{tikzcd}
    X \times_{X\times X} X^I \arrow[r, "\bar{f}"] \arrow[d, "q"]
& X^I \arrow[d, "\pi_X"] \\
 X \arrow[r, "f"]
& X\times X
    \end{tikzcd}
\end{center}
If we choose $f(x) = (x_0,x)$ then we get that 
\eq{
X \times_{X\times X} X^I &= \pbrace{(x,\gamma) \, \rvert \, (x_0,x) = (\gamma(0),\gamma(1)), \, \gamma\in X^I} \\
&= \pbrace{(\gamma(1),\gamma) \, \rvert \, \gamma \in PX}
}
As such we can see that $q : X \times_{X\times X} X^I \to X$ restricts exactly to $p_X$, and so we are done.
\end{proof}

%%%%%%%%%%%%%%%%%%%%%%%%%%%%%%%%%%%%%%%%%%%
\subsection{Part iv)}
We now want to  make the following claim:
\begin{prop}
$PX$ is contractible.
\end{prop}
\begin{proof}
We provide a direct homotopy equivalent $PX \simeq \pbrace{c_{0}}$ the constant path at $x_0$. This is constructed as such: note we have the natural maps $PX \overset{c}{\to} \pbrace{c_{0}}$ and $\pbrace{c_{0}} \xhookrightarrow{i} PX$. $i \circ c = \id_\ast$ so we don't need no show any homotopy here. The content of the proof is then in showing 
\eq{
c \circ i \simeq \id_{PX}
}
We will construct this homotopy as such. 
\eq{
H: PX \times I &\to PX \\
(\gamma,s) &\mapsto \gamma_s 
}
where 
\eq{
\gamma_s(t) = \gamma(ts)
}
\end{proof}

%%%%%%%%%%%%%%%%%%%%%%%%%%%%%%%%%%%%%%%%%%%
\subsection{Part v)}
Let us now think of the fibres of the map $p_X$ over $x$, which are paths in $X$ with endpoints $x_0,x$. It is important at this point to note that $c_0$ is part of the fibre over $x_0$. Now recall we can define the $n$th homotopy group by 
\eq{
\pi_n(X,x_0) = \psquare{(I^n,\del I^n),(X,x_0)}
}
Hence as we have that any map $I^n \to X$ can be written as $I^{n-1}\times I \to X$ or in other words as a map $I^{n-1} \to X^I$, the condition $\del I^n \mapsto x_0$ is equivalent to $\del I^{n-1} \mapsto c_0$, and so 
\eq{
\pi_n(X,x_0) \cong \pi_{n-1}(F,c_0)
}

\end{document}