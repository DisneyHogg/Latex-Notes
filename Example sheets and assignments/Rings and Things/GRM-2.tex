\documentclass{article}

\usepackage{header}
%%%%%%%%%%%%%%%%%%%%%%%%%%%%%%%%%%%%%%%%%%%%%%%%%%%%%%%%
%Preamble

\title{Groups Rings and Modules Example Sheet 2}
\author{Linden Disney-Hogg}
\date{December 2019}

%%%%%%%%%%%%%%%%%%%%%%%%%%%%%%%%%%%%%%%%%%%%%%%%%%%%%%%%
%%%%%%%%%%%%%%%%%%%%%%%%%%%%%%%%%%%%%%%%%%%%%%%%%%%%%%%%
\begin{document}

\maketitle
\tableofcontents

%%%%%%%%%%%%%%%%%%%%%%%%%%%%%%%%%%%%%%%%%%
%%%%%%%%%%%%%%%%%%%%%%%%%%%%%%%%%%%%%%%%%%
\section{Definitions and Theorems}
We will required the following definitions and theorems in order to do our questions: Let $R$ be our ring

\begin{definition}
The \bam{radical} of an ideal $I \vartriangleleft R$ is 
\eq{
\sqrt{I} \equiv \pbrace{ r \in R \, | \, \exists n \in \mbb{N} \text{ s.t. } r^n \in I} \,.
}
\end{definition}

\begin{definition}
The \bam{Jacobson radical} $\mc{J}(R)$ is the intersection of all maximal ideals of $R$. 
\end{definition}

\begin{theorem}
Every non-unit is contained in a maximal ideal
\end{theorem}

\begin{theorem}[Nullstellensatz]
Let $K$ be a field, $R = K[x_1, \dots x_n]$, and $J \vartriangleleft R$. Then 
\eq{
I(V(J)) = \sqrt{J}
}
where
\eq{
V(J) &= \pbrace{a \in K^n \, | \, \forall f \in R , \, f(a)=0 } \\
I(V) &= \pbrace{f \in R \, | \, \forall a \in V , \, f(a)=0 }
}
\end{theorem}
%%%%%%%%%%%%%%%%%%%%%%%%%%%%%%%%%%%%%%%%%%
%%%%%%%%%%%%%%%%%%%%%%%%%%%%%%%%%%%%%%%%%%
\section{Question 2}
%%%%%%%%%%%%%%%%%%%%%%%%%%%%%%%%%%%%%%%%%%
\subsection{(a)}
\begin{prop}
\eq{
r \in \mc{J}(R) \Leftrightarrow \forall s \in R , \, 1+rs \text{ a unit}
}
\end{prop}
\begin{proof}
($\Leftarrow$:) Suppose $\exists M \vartriangleleft R$ a maximal ideal that doesn't contain $r$. Let $S$ be a generating set for $M$. Then as $M$ is maximal, $R = (S,r)$. Specifically, 
\eq{
1 &= ra + \sum_{i=1}^n a_i s_i \\
\Rightarrow 1-ra &= \sum_{i=1}^n a_i s_i
}
for some $a,a_1, \dots,a_n \in R$. Then as $1-ra$ is a unit (take $s=-a$), $M$ contains a unit $\Rightarrow M=R$, contradicting the existence of such a maximal $M$, so $r \in \mc{J}$ \\
($\Rightarrow$:) $r \in \mc{J} \Rightarrow \forall s \in R, \, -rs \in \mc{J}$ as $\mc{J}$ is an ideal. Hence for $M \vartriangleleft R$ a maximal ideal, $1 \notin M \Rightarrow 1+rs \notin M$. Hence as $1+rs$ is contained in no maximal ideals, it must be a unit. 
\end{proof}

%%%%%%%%%%%%%%%%%%%%%%%%%%%%%%%%%%%%%%%%%%
\subsection{(b)}

\begin{prop}
$r$ nilpotent $\Rightarrow 1+r$ a unit 
\end{prop}
\begin{proof}
Let $n \in \mbb{N}$ be such that $r^n = 0$. Then 
\eq{
(1+r)(1-r+r^2 - \dots + (-1)^{n-1}r^{n-1}) &= 1 + (-1)^n r^n = 1
}
so we have found the multiplicative inverse of $r$. 
\end{proof}

\begin{corollary}
All nilpotent elements of $R$ are contained in $\mc{J}$.
\end{corollary}
\begin{proof}
Note if $r\in R$ is nilpotent as above, then $\forall s \in R$ so is $rs$ as 
\eq{
(rs)^n = r^n s^n = 0 \cdot s^n = 0
}
As such $1+rs$ is invertible, and so $r \in \mc{J}$. 
\end{proof}


%%%%%%%%%%%%%%%%%%%%%%%%%%%%%%%%%%%%%%%%%%
%%%%%%%%%%%%%%%%%%%%%%%%%%%%%%%%%%%%%%%%%%
\section{Question 4}
Take the ring $R = \mbb{C}[x,y,z]$.

%%%%%%%%%%%%%%%%%%%%%%%%%%%%%%%%%%%%%%%%%%
\subsection{(a)}

Let $J = (x^2,xy) \vartriangleleft R$. We then have 
\eq{
V(J) = \pbrace{(0,y,z)}
}
and so 
\eq{
\sqrt{I} = (x)
}
$V(J)$ is exactly the $x=0$ plane. 
%%%%%%%%%%%%%%%%%%%%%%%%%%%%%%%%%%%%%%%%%%
\subsection{(b)}

Let $J = (xz,yz) \vartriangleleft R$. We then have 
\eq{
V(J) = \pbrace{(0,0,z), (x,y,0)}
}
We see from this that for a function to be 0 on both these values if must have a factor of both $z$ and either $x$ or $y$. This means 
\eq{
\sqrt{J} = J 
}
precisely. $V(J)$ is the union of the $z=0$ plane and the $z$-axis. 

%%%%%%%%%%%%%%%%%%%%%%%%%%%%%%%%%%%%%%%%%%
\subsection{(c)}
Let $J = (y^2 - 2x^2y + x^4) \vartriangleleft R$. Spotting the factorisation 
\eq{
y^2 - 2x^2y + x^4 = (y-x^2)^2 
}
and knowing $y-x^2$ is irreducible, we have 
\eq{
\sqrt{J} = (y-x^2)
}
The corresponding variety is 
\eq{
V(J) = \pbrace{y=x^2} = \pbrace{(x,x^2,z)}
}

\end{document}